\section{Introduction}

%%%% learning objectives %%%%
% We can explain how VC generation works, and how it can be used for automatically proving certains properties about programs
% Can take a semantic, in form of inference rules, and translate it into concise and correct Haskell code
% Can write modular Haskell code that uses monad transformer stack to enable good seperation of concerns
% Attempted to support our claim of the correctness of our program by using automated testing
% The goal was to implement an assertion language that can be extended with the PER logic

%  -------------------------------

In this project we present a small imperative \textit{While} language in the C-style family with a built-in assertion language.
This language uses Hoare logic and predicate tranformer semantics to allow for verification condition generation similar to \textit{Why3}.
Our VC generator discharges verification conditions fit for the SMT solver \textit{Z3}, allowing us to prove the partial or total correctness of a program.

The main motivation for implementing this language is to make a base which we can easily extend to include \textit{Partial Equivalence Relation} (PER) logic, which is a logic proposed to reason about \textit{Information Flow Control}\cite{Filinski-Jensen-Larsen:Unpublished-IFC}. This can be used for making assertions about the security of code.

\textit{Predicate Transformer Semantics}, which is used for verification condition generation, was introduced by Dijkstra in 1975\cite{Dijkstra}.
These conditions are based on the work of Robert W. Floyd, who presented a form of program verification using logical assertions in his paper from 1967\cite{Floyd1967}. 
Floyds work led to the axiomatic system presented by Sir C.A.R Hoare in 1969\cite{Hoare}, also known as Floyd-Hoare logic. Using this system, and the rules of inference that it introduces, one can formally prove the correctness of a program.
Combined, this provides an approach for automatically generating conditions which can verify the correctness of certain programs.
% topic
% When writing code, it is important to verify that a program does in fact do as intended.
% Often this is done by extensive testing, but this might not be sufficient to assert correctness of said program.
% We take our starting point in the work of Floyd, Hoare and Dijkstra.
% In 1969 C.A.R. Hoare puplished a paper on an axiomatic system for proving correctness of programs, which laid the base for Hoare logic, a system for reasoning about program logic\cite{Hoare}.
% Using this logic, and the rules of inference that it introduces, one can formally prove the correctness of a program.
% Floyd did something similar in his paper from 1967, but by using flow-diagrams, rather than an axiomatic system\cite{Floyd1967}.
% % However, doing this by hand is time consuming, and can be expensive.

% In 1975, Dijkstra introduced \textit{predicate transformer semantics}, which is a way to find a suiting verification condition for a program, based on assertions about said program\cite{Dijkstra}.
% This provides an approach for automatically generating conditions which we can use to verify the correctness of certain programs.
% % noget med information flow control??

% % hvad er konteksten for problemet

% % hvad er der blevet gjort før?
% % there are several programming languages allowing for verification condtition generation etc...
% % Z3 why3
\textit{Why3} is an example of a platform which provides automatic program verification.
\textit{Why3} requires the programmer to write assertions inside the program code, and then utilises these assertions to compute verification conditions for the program.
Once the verification condition is generated, \textit{Why3} can discharge this condition to a variety of SMT solvers or proof assistants, which can then determine whether this condition can be proved.
% Thus \textit{WhyML} is a programming language with an assertion interface, which allows automatic verification of programs by external SMT-solvers.

% vores løsningsforslag

% hvad er vores mål med projektet
% IFC can help us assert that variables are safe...
This report presents the work of this project as follows:

% scope and limitations to the solution
\begin{itemize}
  \item In \cref{sec:Background}, we present the syntax and semantics of our language.
  Next we explain how VC generation works, how it is related to Hoare logic and how it can be used for automatically proving certains properties about programs. Finally we describe the coupling between VC-generators and SMT-solvers.
\item In \cref{sec:implementation}, we present our implementation, which can both dynamically evaluate a program and statically prove the correctness.
In the implementation we transform the formal semantics of the language into equivalent Haskell code and utilize predicate transformer semantics to compute verification conditions for a program.
\item In \cref{sec:assessment}, we present our strategy for evaluting the code quality and give an assessment of the implementation.
 Here, a combination of blackbox tests, automatic tests, and test programs, is used to assess the quality.
% Here we will also reflect on the limitations modern SMT-solvers.
\item Lastly, \cref{sec:conclusion} presents some ideas for future work, gives a discussion of the work, and concludes the project.
\end{itemize}

Throughout the project our main focus has been on getting a good understanding of how verification conditions work, and how to convert this into Haskell code. It has also been on building an implementation that is robust and easily extended and maintained.
Furthermore it has been interesting for us to investigate what is necessary to prove correctness of a program, using our implementation, and how to thouroughly test the correctness of our work.

Enjoy!

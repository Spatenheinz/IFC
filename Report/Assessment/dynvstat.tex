Besides the QuickCheck and blackbox testing, we experimented with the example programs to check the quality of our implementation.
In this subsection we present some of the example programs written for testing, and describe how and why they are interesting. In this process we relate the programs to equivalent to \textit{Why3} programs, to determine the expressiveness of our language.
Next we set forth the experiments that we conduct using the programs as input to both the \textit{Interpreter} and the \textit{VC generator}.
Finally we explain how we compare dynamic execution to static verification of the programs.

\paragraph{Example programs.}
% What example programs have we written and why? (some examples)
In \cref{sec:testprograms} we present an overview of some of the example programs written to test the implementation.
% simple.ifc???
The program \texttt{always\_wrong} tests the \texttt{violate} construct, and simply checks whether the program correctly fails.
The \texttt{skip} program tests the \texttt{skip} command, by asserting that the store is unchanged after executing the command.
The programs \texttt{assign} and \texttt{max} are very simple too, and computes the result without use of while-loops, thus testing simple statement constructs.

The more interesting examples uses while-loops, in which the invariants and variants are important. The following present some of them, what they test, and why they are interesting.

\begin{itemize}
  \item {\texttt{mult.ifc}.} 
  % the running example. Shows both assignment, while-loop, assertions, and ghost variables, and also with input vars and requirements.
	The program takes as input two variables $q,r$ and computes $q \times r$.
  This is the example that we use throughout the report, because it showcases a sequence of statements including assignments, assertions, use of ghost variables, and while loops with both an invariant and a variant, and is still easy to follow.
  Thus the program tests a simple combination of statements, including a while loop that always terminates, and therefore the program should be provable under total correctness.
  An interesting thing about the program is investigating whether the program is provable with the given invariant and variant.
  The code for \texttt{mult.ifc} is shown in \cref{figure:mult}. Furthermore we made a modified program \texttt{mult3.ifc} which multiplies 3 variables by addition and hence tests the usage of nested while loops.

  \item{\texttt{collatz.ifc}.}
  % something about termination
  This program calculates the collatz sequence of the input variable $n$. The while loop will run until $n = 1$.
  The idea behind this program is to have a while loop for which we cannot provide a variant. 
%	\begin{lstlisting}
%	while (n /= 1) ?{ n > 0}{
%	      if (n % 2 = 0) {
%	         n := n / 2;
%	      } else {
%	         n := 3 * n + 1;
%	      };
%	};
%	if n = 1 { k := 42; };
%	#{ k = 42};
%	\end{lstlisting}
	We can prove partial correctness of the program, but cannot prove total correctness.
	Ie., if we assume that the while loop terminates, we know that $n$ is indeed $1$ at loop termination, but we can never prove that the loop terminates.
  The program utilises assertions in a way that ensures that the postconditon only holds assuming the loop terminates.
  Therefore, a loop variant is necessary to determine whether the program is correct or not.

  \item{\texttt{undef.ifc}.}
	The programs \texttt{undef1.ifc}, \texttt{undef2.ifc} and \texttt{undef3.ifc} are examples of negative testing.
  They each test different undefined behaviour:
	\begin{enumerate}
		\item Testing that modulo by zero is undefined behaviour, and that the leftmost error is output.
		\item Testing that using an undeclared variable is undefined behaviour, and that the leftmost error is output.
		\item Testing that the leftmost error is output.
	\end{enumerate}
  Running the program shows how modulo and division by zero, along with using undeclared variables, is undefined behaviour.
	Furthermore, it tests that we always output the leftmost error.

	These programs also show how use of undefined behaviour is considered a user error, thus trying to prove the programs results in falsifiable counter examples, as undefined behaviour implies \textit{false}.

  \item{\texttt{mccarthy.ifc}.}
   This program calculates the McCarthy 91 function. It takes as input an integer $x$ and computes the result according to the formula
	\[ f(x) =
	  \begin{cases}
	    x - 10       & \quad \text{if } x \ge 100 \\
	    f(f(x + 11))  & \quad \text{otherwise}
	  \end{cases}
  \]
  The result can also be described as:
	\[ f(x) =
	  \begin{cases}
	    91       & \quad \text{if } x \leq 100 \\
	    x - 10   & \quad \text{otherwise}
	  \end{cases}
	\]
  The interesting thing about this function is that although we can express the nested recursion as a while loop and this function indeed is correct, we have no way of expressing a valid variant and thus cannot provide strong enough invariants to prove the program.
        For us to be able to do this, we need to allow for pairs in the variant and to have recursive function calls in assertions.
        Hence, we have a case of a problem which shows the relative completeness of Hoare logic, as we cannot prove our program although it is in fact correct. We can ensure so by comparing with an equivalent Why3 program.
  \item{\texttt{isqrt.ifc, isqrt\_sub.ifc \& isqrt\_fast.ifc}.}
        \texttt{isqrt.ifc} calculates the integer square root of a non-negative integer. We can both evaluate this in the interpreter, and prove correctness of the program.
        \texttt{isqrt\_sub.ifc} calculates the integer squareroot using subtraction, and is also included in the tests.
        We further looked at an algorithm from Why3's verified programs gallery, which uses Newtons method for calculating this. This is the \texttt{isqrt\_fast.ifc} program.
        Unfortunately we are not able to prove this, although we can write a semantically equivalent program. The reason for this is that Why3 needs to some transformations of the subgoals for the proof to even validate it. Our solution is not able to do any such transformations.
\end{itemize}

\paragraph{Experiments with example programs.}
We have automated tests for testing that the \textit{Interpreter} can correctly evaluate a program,
and for testing that the programs that are provable using the VC generator will also evaluate to \textit{true} in the \textit{Interpreter}.

% testing the evaluator by running program with evaluator and comparing result to haskell result
The first type of test, testing the evaluation of programs, is done by generating random input for the example programs, and then asserting that the result is in fact what we expect.
For example, when evaluating the \texttt{mult.ifc} program with two random values, the result should be equal to the result of multiplying the two values in Haskell.
It should be noted that we use generators for generating meaningful input to the programs, to be able to test with all kinds of valid input.

% comparing running programs through evaluator and vc generator
The second type of test asserts that provably correct programs will evaluate correctly as well.
This is tested by first feeding the programs to the VC generator coupled with Z3, and then to the \textit{Interpreter} with random input.
Given that the program is provable the \textit{Interpreter} evaluates all assertions to \textit{true}.
Note that if we only show partial correctness, the \textit{Interpreter} might run forever, so we only test for terminating instances.
If the program is falsified, then the test will run the program with the falsifiable instance and assert that the \textit{Interpreter} will terminate abnormally.

% TODO: checking with Why3
Another method that we use for verifying the correctness of our implementation is to check whether our application produces a result equivalent to that of \textit{Why3}.
Because \textit{Why3} can use the same solver (\textit{Z3}), and the same theories, we can create \textit{Why3} programs equvalent to our test programs, and assert that they both either succeed in proving correctness, or fail with a counter example.
Furthermore we have both positive and negative test programs, testing that our implementation both succeeds when it should, and fails when a program cannot be proven correct.

\paragraph{Provable by VC generation ensures successful evaluation.}
% Provable => true in evaluation, but not necessarily the other way around
% Here we want to address how the fact that a program is provable with c generation and SMT solving means that the program will also evaluate corretly, but that it is not necessarily the case the other way around.

As described, we value consistency highly, and it should always hold true that if we can prove total correctness of a program, then the dynamic evaluation should give the expected result. Once again it is important to note that this will only hold for total correctness and not necessarily for partial correctness.
% As described above, an important property on the relation between the dynamic execution of a program through the evaluator, and the static proof of said program, is that if we can correctly prove the correctness of a program, then the dynamic evaluation should also hold.
Oppositely, correct dynamic evaluation does not imply provable correctness.
The reason for this is that we might not have provided strong enough assertions to generate an appropriate verification condition, whilst the dynamic execution just needs all assertions to evaluate to \textit{true}. 
This might be even more apparent, considering that quantifiers does not work correctly in the \textit{Interpreter}.
If for example we have a program that uses \textit{true} as a loop invariant, this will hold in each iteration during the dynamic execution, but will probably not be enough to prove any postcondition statically.

Lets take a closer look at the multiplication example program from \cref{figure:mult}. 
% \lstinputlisting{Examplecode/mult.ifc}
% \begin{lstlisting}
% vars: [q,r]
% requirements: {q >= 0 /\ r >= 0}
% <!=_=!>
% res := 0;
% $a := q;
% while (q > 0) ?{res = ($a - q) * r /\ q >= 0} !{q} {
%       res := res + r;
%       q := q - 1;
% };
% #{res = $a * r};
% \end{lstlisting}
We have previously argued that the code is correct and can correctly be proved by Z3, but if we relax some of the assertions in the program, this will no longer be the case.
Consider exchanging the loop-invariant 
\begin{verbatim}
?{res = ($a - q) * r /\ q >= 0}
\end{verbatim}
 with the looser invariant 
\begin{verbatim}
?{res = ($a - q) * r}
\end{verbatim}
Now Z3 will no longer be able to prove the correctness of the program. The generated formula looks as shown in \cref{fig:modmult}.

\begin{figure}
\begin{lstlisting}[mathescape=true]
$\forall q, r. \; (q \geq 0 \land r \geq 0) \Rightarrow$
$ \quad \forall res_{3}. \; res_{3} = 0 \Rightarrow$
$\quad \quad \forall \$a. \; \$a = q \Rightarrow$
$\quad \quad \quad (res_{3} = (\$a - q) * r \land q \geq 0)$
$\quad \quad \quad \; \land \forall q_{2}, res_{2}, \xi_{1}.$
$\quad \quad \quad \quad (q_{2} > 0 \land res_{2} = (\$a - q_{2}) * r \land \xi_{1} = q_{2}) \Rightarrow$
$\quad \quad \quad \quad \quad \forall res_{1}. \; res_{1} = res_{2} + r \Rightarrow$
$\quad \quad \quad \quad \quad \quad \forall q_{1}. \, q_{1} = q_{2} - 1 \Rightarrow$
$\quad \quad \quad \quad \quad \quad \quad (res_{1} = (\$a - q_{1}) * r \land 0 \le \xi_{1} \land q_{1} < \xi_{1})$
$\quad \quad \quad \land ((q_{2} \leq 0 \land res_{2} = (\$a - q_{2}) * r) \Rightarrow res_{2} = \$a * r)$
\end{lstlisting}
\caption{Generated verification condition for the modified \texttt{mult} program.}
\label{fig:modmult}
\end{figure}

From the formula it becomes quite apparent that the invariant is no longer strong enough to prove the condition, since the restriction on $q_{2}$ is too weak. 
Z3 gives us a falsifiable example where $q_2 = -3$, $res_2 = 6$, $\$a = 0$ and $r=2$.
Using these values, the two first conjunctions in line $4$ and lines $5$-$9$ will evaluate to \textit{true}.
In the last term in line $10$, the LHS of the implication will evaluate to \textit{true} as
$(-3 \leq 0 \land 6 = 3 * 2)$
is correct, but the RHS will evaluate to \textit{false}, since
$6 \neq 0 * 2$
is not correct. Thus the counter example does in fact falsify the verification condition.

But we know from testing that the program does in fact behave as intended, so how can the prover falsifiy the formula? 
We can verify that our application acts correctly, by doing the same modification to the whyML program in \cref{fig:why3}.
Trying to prove the program correctness gives a falsifiable counter example, as expected.
By this, we have confidence that our implementation works correctly, and requires the necessary loop invariants.
% \\~\\


% We have conducted a smaller variety of tests. The tests includes both some testing of properties thorugh QuickCheck along with some examples program.

We have conducted a variety of tests to assess the implementation, consisting of both automatic tests and blackbox tests, and some manual experimentation using Why3.
To thoroughly test our implementation, the tests follow this testing strategy:
\begin{itemize}
	\item The \textit{Parser} is tested using a blackbox approach, which should cover all aspects of the grammar presented in \cref{table:grammar}.
  Furthermore, we have a property based QuickCheck test to test for larger program constructs.
	\item The test suite includes property based testing to test the semantics of the \textit{While} language, which compares the result of running the \textit{Interpreter} on two semantically equivalent generated program constructs.
	\item We attempt to use property based testing to compare interpretation of generated programs with the result of using our VC-generator coupled with \textit{Z3} on the same program. This part is lacking since we want to generate some meaningful building blocks, which showed to be quite challenging, and therefore we did not succeed, as this was not a critical part of this project.
  \item We use QuickCheck to generate random input for our example programs, which we use to run the programs and compare with the result of running the same computation in Haskell. This is to assert that the example programs do indeed behave as we expect them to, indicating whether they should be provable or not.
	\item We test that a provable verification condition also results in a correct dynamic evaluation.
  \item To support the automated tests, we have done some analysis of a selection of programs and compared them to equivalent \textit{Why3} programs.
\end{itemize}

A substantial part of the test suites and the experimentation is based on example programs, which can be found in \cref{sec:exampleprogs}.
Some of the examples are working programs, and some are designed to fail.
\cref{sec:testprograms} presents the programs (leftmost column), a short description of what they do (middle column), and the results of running them through the \textit{Interpreter} and \textit{VC generator} (rightmost column).
We will analyse a selection of the examples in \cref{sec:examples}.

In \crefrange{sec:blackbox}{sec:examples} we present the testing strategy more thoroughly.
First we present the strategy of our blackbox \textit{Parser} tests. 
Next we present our approach to generating automatic tests with property based testing. 
Finally we go into details with the example programs, and how they are designed to test the functionality of the implementation.

\subsubsection{Blackbox testing}\label{sec:blackbox}
% blackbox tetsing of parser


\subsubsection{Quickchecking instances}\label{sec:qc}
% To further test our implementation, we design a test suite of property-based tests. Mostly these tests ensure that the evaluator follows the semantics.
To further test certain properties of the implementation, we build a test suite consisting of property based tests.
With these automatic tests we attempt to verify that
\begin{itemize}
	\item That the \textit{Parser} correctly parses larger program constructs.
	\item That the \textit{Evaluator} correctly follows the semantics of the \textit{While} language. 
	\item That the result of running a program through the \textit{VC generator} is equivalent to running it through the \textit{Evaluator}.
\end{itemize}
To do this, we build generators for generating suitable input for the tests, and next we find appropriate properties to test using input from the generators.

\paragraph{Generating input.}
TODO: skriv dette færdigt.
% beskrivelse af hvordan vi genererer ASTs
% laver nogle generators for forskellige constructs, og sætter dem sammen med frequency
% while som eksempel
% meaningfulness af det genererede input
Using the QuickCheck interface, we define instances of \textit{Arbitrary} for the different AST types.
When doing this, there are certain important considerations.
%To enable this, we define an Arbitrary instance on our AST types. Most of these are quite generic, however there are certain considerations that is quite important.
Firstly, to ensure that the size of the generated expressions do not explode, we use \textit{sized expressions} to control expansion, and to ensure that we get a good distribution of the various constructs we use \textit{frequency} to choose between them.
Secondly, we want the number of possible variables to be limited, such that the \textit{Evaluator} will not fail too often, by using variables that have not yet been defined. This is done by limiting the options for variable names to be only single character string.
Thirdly, while-loops might not terminate, hence we want to define a small subset, or a skeleton, for while-loops that we can be sure terminates. 

To accomodate this third consideration, we build a skeleton for while-loops that should always terminate. This is done by defining three versions of while-loops as shown in \autoref{fig:genwhile}.
TODO: put in correct code!!!
\begin{figure}
\begin{lstlisting}
whileConds = elements $ zip3 (replicate 4 ass) [gt, lt, eq] [dec, inc, change]
  where v = whilenames
        v' = Var <$> v
        ass = liftM2 Assign v arbitrary
        gt = liftM2 (RBinary Greater) v' arbitrary
        dec = liftM2 Assign v (liftM2 (ABinary Sub) v' (return $ IntConst 1))
        lt = liftM2 (RBinary Less) v' arbitrary
        inc = liftM2 Assign v (liftM2 (ABinary Add) v' (return $ IntConst 1))
        eq = liftM2 (RBinary Eq) v' arbitrary
        change = liftM2 Assign v arbitrary
\end{lstlisting}
\caption{Generation of while-loops using a skeleton.}
\label{fig:genwhile}
\end{figure}

TODO: koden virker ikke lige nu, vi kommer tilbage til det snart :))
%We generate an arbitrary variable, which we ensure will never clash with any of the variables that are not possible to generate elsewhere. The constructs will then be one of the following:
%\begin{itemize}
%  \item Bolean condition is $v > a$, where $v$ is the variable and $a$ is an arbitrary arithmetic expression. Inside the while-loop, we will ensure to decrement $v$, eventually terminating the loop.
%  \item Bolean condition is $v < a$. Inside the while-loop, we will ensure to increment $v$, eventually terminating the loop.
%\item Bolean condition is $v = a$. Inside the while-loop, we will ensure to change $v$, eventually terminating the loop.
%\end{itemize}

% TODO: er det meaningful?

\paragraph{QuickCheck properties.}
Now that we have investigated how to generate input for the tests, we move on to finding meaningful properties to test. In this test suite we use property based testing for the following:
% hvad bruger vi det til
%  - parser test
%  - semantics
%  - equivalence
\begin{itemize}
	\item \textbf{Parser tests.}
	To complement the blackbox tests for the parser, we use automatic testing to generate ASTs and assert that these programs are parsed correctly. This is done by generating an AST, and then using a pretty printer to convert it into a program that can be given as input to the parser. It is then checked whether the result of parsing this program is equal to the original generated AST. This is supposed to assert that the parser can handle a lot of different combinations of constructs, potentially finding bugs that would not have been discovered through our systematic blackbox testing of simple constructs.
	\item \textbf{Semantic equivalence.}
  To test whether the evaluator corretly implements the semantics of the \textit{While} language, we have tested certain equivalence properties. From studying the semantic system for \textit{While}, we have designed equivalence properties according to the semantics. 
	Examples of such equivalence properties are \texttt{if true then s1 else s2} $\sim$ \texttt{s1} and \texttt{while false do s} $\sim$ \texttt{skip}.
  These are designed to cover all the cases of the small-step semantics, and the properties uses generators for generating suitable input variables and statements, i.e. the body of an $if$-statement is generated automatically, but the equivalence relation is defined manually.
	\item \textbf{Evaluating a program vs solving with VC generator and Z3.}
	When given an input program, it should always be true that if the \textit{VC generator} computes a provable verification condition, then the \textit{Evaluator} will evaluate all assertions in the program to \textit{true}.
	To test this, we generate random programs and assert that if the \textit{VC generator} can verify the program, then the \textit{Evaluator} will evaluate to \textit{true}. 
	Here we use generators to generate input for the test. 
	However, testing this automatically is quite a complex situation, since we need to be able to have a strong enough loop-invariant to prove the correctness of the program. 
	It has come to our attention that the generated programs are not very useful, and currently we have not been succesful in implementing such a property-test. We will come back to this in the assessment in \autoref{sec:evaluation}.
	It should also be noted that this is only true in one direction. We discuss why in \autoref{sec:examples}.
%	We test that whenever we can prove the correctness of an IFC program, the result will be correct. 
	%Ideally we would also want to be able to generate programs and test this using QuickCheck. 
\end{itemize}

The above bulletpoints presents the intuition behind the QuickCheck testing of the implementation. A presentation of the test results and assessment of the code will be given in \autoref{sec:evaluation}.
% TODO: presentation of (some important) properties as examples.
























\subsubsection{Dynamic execution compared to static proofs}\label{sec:examples}
Besides the QuickCheck and blackbox testing, we use example programs to check the quality of our implementation.
In this subsection we present some of the example programs written for testing, and describe how and why they are interesting.
Next we set forth the experiments that we conduct using the programs as input to both the \textit{Evaluator} and the \textit{VC generator}.
Finally we explain how we compare dynamic execution to static verification of the programs.

\paragraph{Example programs.}
% What example programs have we written and why? (some examples)
% TODO: comment on the necessity of variants. We did not find programs that actually need a variant.
NOTE: Here we want to include a presentation of our strategy for writing test programs, and maybe present one or two of the most interesting ones.


\paragraph{Experiments with example programs.}
% TODO: think about the difference between a program with insufficient assertions resulting in a falsifiable verification condition, and a wrongly generated verification condition that is (mistakenly) falsifiable.
NOTE: Here we will want to write about the experiments conducted using the example programs, fx comparing the result of running programs through both the evaluator and vc generator.


\paragraph{Provable with VC generation ensures successful evaluation.}
% Provable => true in evaluation, but not necessarily the other way around
NOTE: Here we want to address how the fact that a program is provable with c generation and SMT solving means that the program will also evaluate corretly, but that it is not necessarily the case the other way around.

Another important property on the relation between the dynamic execution of a program via the evaluator and the static proof of said program, is that if we can correctly prove the correctness of a program, then the dynamic evaluation should also hold. It is important to note that the implication does not hold in the other direction. The reason for this is that we might not have provided strong enough assertions to satisfy the generated formula, whilst the dynamic execution, might not need it to be correct.

If we take a closer look at the multiplication example previously.
% \lstinputlisting{Examplecode/mult.ifc}
\begin{lstlisting}
vars: [q,r]
requirements: {q >= 0 /\ r >= 0}
<!=_=!>
res := 0;
$a := q;
while (q > 0) ?{res = ($a - q) * r /\ q >= 0} !{q} {
      res := res + r;
      q := q - 1;
};
#{res = $a * r};
\end{lstlisting}
We have previously argued that the code is correct and can correctly be proved by Z3, but if we relax some of the assertions in the program, this will no longer be the case.
Instead of the loop-invariant \verb|?{res = ($a - q) * r /\ q >= 0}| consider the invariant \verb\?{res = ($a - q) * r}\ in this case Z3 will no longer be able to prove the correctness of said program. The generated formula looks as follows:
\begin{lstlisting}[mathescape=true]
$\forall q, r. \; (q \geq 0 \land r \geq 0) \Rightarrow$
$ \quad \forall res_{3}. \; res_{3} = 0 \Rightarrow$
$\quad \quad \forall \$a. \; \$a = q \Rightarrow$
$\quad \quad \quad (res_{3} = (\$a - q) * r \land q \geq 0)$
$\quad \quad \quad \; \land \forall q_{2}, res_{2}, \xi_{1}.$
$\quad \quad \quad \quad (q_{2} > 0 \land res_{2} = (\$a - q_{2}) * r \land \xi_{1} = q_{2}) \Rightarrow$
$\quad \quad \quad \quad \quad \forall res_{1}. \; res_{1} = res_{2} + r \Rightarrow$
$\quad \quad \quad \quad \quad \quad \forall q_{1}. \, q_{1} = q_{2} - 1 \Rightarrow$
$\quad \quad \quad \quad \quad \quad \quad (res_{1} = (\$a - q_{1}) * r \land 0 <= \xi_{1} \land q_{1} < \xi_{1})$
$\quad \quad \quad \land ((q_{2} \leq 0 \land res_{2} = (\$a - q_{2}) * r) \Rightarrow res_{2} = \$a * r)$
\end{lstlisting}
It becomes quite apparant that the invariant is no longer strong enough to be proved, since the restriction on $q_{2}$ is too weak. The two first conjuctions line 4 and line 5-9, will be true, given that $q_{2} = -3, res_{2} = 6, \$a = 0, r = 2 $.
In the last term
$(q_{2} \leq 0 \land res_{2} = (\$a - q_{2}) * r) \Rightarrow res_{2} = \$a * r$
, the RHS of the implication, will be true as
$(-3 \leq 0 \land 6 = 3 * 2)$, giving a false term.
Once again we can verify that our program acts correctly, by doing the same modification to the whyML program in \autoref{fig:whyml}. In this case the proof will as expected give a falsifiable counter-example. By this, we have a good certainty that our implementation is correct.
\\~\\
We test that whenever we can prove the correctness of an IFC program, the result will be correct. But ideally we would also want to be able to generate programs and test this using QuickCheck. However this is quite a complex situation, since we need to be able to have a strong enough loop-invariant to prove the correctness of the program. As of yet, we have not been succesful in implementing such a property-test.



% We have conducted a smaller variety of tests. The tests includes both some testing of properties thorugh QuickCheck along with some examples program.

We have conducted a variety of tests to assess the implementation, consisting of both automatic tests and blackbox tests, and some manual experimentation using Why3.
To thoroughly test our implementation, the tests follow this testing strategy:
\begin{itemize}
	\item The parser is tested using a blackbox approach, which should cover all aspects of the grammar presented in \cref{table:grammar}.
  Furthermore, we have a property based QuickCheck test to test for larger program constructs.
	\item The test suite includes property based testing to test the semantics of the \textit{While} language, which compares the result of running the interpreter on two semantically equivalent generated program constructs.
	\item We attempt to use property based testing to compare interpretation of generated programs with the result of using our VC-generator coupled with Z3 on the same program. This part is lacking since we want to generate some meaningful building blocks, which showed to be quite challenging, and therefore we did not succeed, as this was not a critical part of this project.
  \item We use QuickCheck to generate random input for our example programs, which we use to run the programs and compare with the result of running the same computation in Haskell. This is to assert that the example programs do indeed behave as we expect them to, indicating whether they should be provable or not.
	\item We test that a provable verification condition also results in a correct dynamic evaluation.
  \item To support the automated tests, we have done some analysis of a selection of programs and compared them to equivalent Why3 programs.
\end{itemize}

A substantial part of the test suites and the experimentation is based on example programs, which can be found in \cref{sec:exampleprogs}.
Some of the examples are working programs, and some are designed to fail.
\cref{sec:testprograms} presents the programs (leftmost column), a short description of what they do (middle column), and the results of running them through the interpreter and VC generator (rightmost column).
We will analyse a selection of the examples in \cref{sec:examples}.

In \crefrange{sec:blackbox}{sec:examples} we present the testing strategy more thoroughly.
First we present the strategy of our blackbox \textit{Parser} tests. 
Next we present our approach to generating automatic tests with property based testing. 
Finally we go into details with the example programs, and how they are designed to test the functionality of the implementation.

\subsubsection{Blackbox testing}\label{sec:blackbox}
% blackbox testing of parser
% Here we will present the strategy of our blackbox parser tests.
To assert that our implementation correctly parses input programs, we have designed a blackbox test suite for systematically testing each construct of the grammar.
The strategy is to test parsing of smaller constructs such as variable names and expressions, and then combine them into larger constructs such as statements.
Taking a very systematical approach we jope to cover all aspects of the grammar.

Furthermore, we use blackbox tests to assert that the \textit{Parser} handles ambuguities, associativity and precedence correctly.

Finally, we have designed both positive and negative nests for the \textit{Parser}, to assert that it behaves as intended when given both good and bad input.

TODO: credit to Andrzej Filinski for nice tests :)


\subsubsection{Quickchecking instances}\label{sec:qc}
% To further test our implementation, we design a test suite of property-based tests. Mostly these tests ensure that the evaluator follows the semantics.
% To further test certain properties of the implementation, we build a test suite consisting of property based tests.
Other than the blackbox tests for the parser, our tests are mostly property based.
With these automatic tests we attempt to verify the following:
\begin{itemize}
	\item That the parser correctly parses larger program constructs.
	\item That the interpreter correctly follows the semantics of the \textit{While} language.
	\item That there is consistency between the \textit{VC-generator} and the interpreter.
\end{itemize}
We build generators for the AST constructs, to enable generation of arbitrary input programs, and then use the generated input to test the properties specified above.

\paragraph{Generating input.}
% beskrivelse af hvordan vi genererer ASTs
% laver nogle generators for forskellige constructs, og sætter dem sammen med frequency
% while som eksempel
% meaningfulness af det genererede input
Using the QuickCheck interface, we define instances of \textit{Arbitrary} for the different AST types.
When doing this, there are certain important considerations.
%To enable this, we define an Arbitrary instance on our AST types. Most of these are quite generic, however there are certain considerations that is quite important.
Firstly, to ensure that the size of the generated expressions does not explode, we use \textit{sized expressions} to control expansion, and to ensure that we get a good distribution of the various constructs we use \textit{frequency} to choose between them.
Secondly, we want the number of possible variables to be limited, such that the interpreter will not fail too often, by using variables that have not yet been declared.
This is done by limiting the options for variable names to be only single character string (this might still be too much).
Thirdly, while loops might not terminate, hence we want to define a small subset, or a skeleton, for while loops that are sure to terminate. 
This task proved quite challenging, since we need to construct while loops that are somewhat meaningful, in that they must terminate, and the invariant must be enough for the program to be validated. We have been able to generate some while-loops but they are not meaningful enough for the verifier, however, they can still be used in the semantic equivalence tests between different program constructs.

\paragraph{QuickCheck properties.}
With the possibility of generating input for the tests, we move on to finding meaningful properties to test. In this test suite we use property based testing for the following:
\begin{itemize}
  \item \textbf{Parser tests.}
  To complement the blackbox tests for the parser, we want the following property to hold: \textit{parse (prettyprint a) = a}, where $a$ is an arbitrary AST. This is supposed to assert that the parser can handle a lot of different combinations of constructs, potentially finding bugs that would not have been discovered through our systematic blackbox testing of simple constructs.
	\item \textbf{Semantic equivalence.}
  To test whether the Interpreter corretly implements the semantics of the \textit{While} language, we have tested certain equivalence properties. From studying the semantic system for \textit{While}, we have designed equivalence properties according to the semantics.
	Examples of such equivalence properties are \texttt{if true then s1 else s2} $\sim$ \texttt{s1} and \texttt{while false do s} $\sim$ \texttt{skip}.
  These relations are directly related to the small-step semantics. 
  The properties use generators for generating suitable input variables and statements, i.e. the body of an $if$-statement is generated automatically, but the equivalence relation is defined manually.
	\item \textbf{Evaluating a program vs solving with VC generator and Z3.}
    To ensure consistency between the static and dynamic evaluation we wanted to generate some arbitrary programs and check for consistency.
	However, testing this automatically is quite a complex situation, since we need to be able to have a strong enough loop-invariant to prove the correctness of the program.
	It has come to our attention that the generated programs are not very useful, and currently we have not been succesful in implementing such a property-test. We will come back to this in the assessment in \cref{sec:evaluation}.
\end{itemize}

The above bullet points presents the intuition behind the QuickCheck testing of the implementation. A presentation of the test results and assessment of the code will be given in \cref{sec:evaluation}.
























\subsubsection{Dynamic execution compared to static proofs}\label{sec:examples}
Besides the QuickCheck and blackbox testing, we experimented with the example programs to check the quality of our implementation.
In this subsection we present some of the example programs written for testing, and describe how and why they are interesting.
Next we set forth the experiments that we conduct using the programs as input to both the \textit{Interpreter} and the \textit{VC generator}.
Finally we explain how we compare dynamic execution to static verification of the programs.

\paragraph{Example programs.}
% What example programs have we written and why? (some examples)
% TODO: comment on the necessity of variants. We did not find programs that actually need a variant.
% Here we want to include a presentation of our strategy for writing test programs, and maybe present one or two of the most interesting ones.
In \cref{table:testprograms} we present an overview of some of the example programs written to test the implementation.
% simple.ifc???
The program \texttt{always\_wrong} tests the \texttt{violate} construct, and simply checks whether the program correctly fails.
The \texttt{skip} program tests the \texttt{skip} command, by asserting that the store is unchanged after executing the command.
The programs \texttt{assign} and \texttt{max} are very simple too, and computes the result without use of while-loops, thus testing simple statement constructs.

The more interesting examples uses while-loops, in which the invariants and variants are important. The following present some of them, what they test, and why they are interesting.

\begin{itemize}
  \item {\texttt{mult.ifc}.} 
  % the running example. Shows both assignment, while-loop, assertions, and ghost variables, and also with input vars and requirements.
  This is the example that we use throughout the report, because it showcases a sequence of statements including assignments, assertions, use of ghost variables, and while-loops with both an invariant and a variant.
  Thus the program tests a simple combination of statements, including a while loop that always terminates, and therefore the program should be provable under total correctness.
  An interesting thing about the program is to investigate whether the program is provable with the given invariant, and whether a variant is needed to prove correctness.
  The code for \texttt{mult.ifc} is shown in \cref{figure:mult}.
  
  \item{\texttt{generic\_sum.ifc}.}\footnote{check dette eksempel igennem igen.}
  % something about generating while-loops?
  The idea behind this program is to make a generic while-loop summing up a list of values. 
  The program takes as input an offset, a step and a limit, and then sums up all the values of a list starting from the offset, stepping with the given value up to the limit.
  This is part of an experiment to test more types of while loops. Provided a loop skeleton, we can generate random statements for the loop body, and in that way test various kinds of while loops that are ensured to terminate.
  However, this program has such a strong loop condition compared the the postcondition, that it will always be provable.
  The goal is to find a loop skeleton which provides just enough information so that the invariant is crucial for proving correctness. If the variant is needed for proving correctness as well, the skeleton is ideal for generating while-loops.
  
  \item{\texttt{collatz.ifc}.}
  % something about termination
  This program calculates the collatz sequence of the input variable $n$. The while loop will run until $n = 1$.
        The idea behind this program is to have a while loop which we cannot provide a variant. Alas this program terminates for all $n > 0$. We can see this by the following:

\begin{minipage}[t]{0.4\textwidth}
  \begin{lstlisting}
while (n /= 1) ?{ n > 0}{
      if (n % 2 = 0) {
         n := n / 2;
      } else {
         n := 3 * n + 1;
      };
};
if n = 1 { k := 42; };
#{ k = 42};
\end{lstlisting}
\end{minipage}
\begin{minipage}[t]{0.4\textwidth}
  \begin{lstlisting}
while (n /= 1) ?{ n > 0}{
      if (n % 2 = 0) {
         n := n / 2;
      } else {
         n := 3 * n + 1;
      };
};
#{false};

\end{lstlisting}
\end{minipage}

  Interestingly enough, the program on the left can be proved, whilst the program on the right cannot. 
  This tells us that the loop must terminate, even without a variant. 
  As the program on the right would be partially correct, given that the loop does not terminate, since any postcondition $Q$ is valid if the command does not terminate.
  % This can tell valuable things about attempting to prove non-terminating programs.
  % The program utilises assertions in a way that ensures that the postconditon only holds if the loop has terminated.
  % Therefore, a loop variant is necessary to determine whether the while-loop terminates or not.
  % TODO: this program is not quite done.
\end{itemize}

\paragraph{Experiments with example programs.}
% TODO: think about the difference between a program with insufficient assertions resulting in a falsifiable verification condition, and a wrongly generated verification condition that is (mistakenly) falsifiable.
We have automated tests for testing that the \textit{Interpreter} can correctly evaluate a program,
and for testing that the programs that are provable using the VC generator will also evaluate to \textit{true} in the \textit{Interpreter}.

% testing the evaluator by running program with evaluator and comparing result to haskell result
The first type of test, testing the evaluation of programs, is done by generating random input for the example programs, and then asserting that the result is in fact what we expect.
For examle, when evaluating the \texttt{mult.ifc} program with two random values, the result should be equal to the result of multiplying the two values in Haskell.
It should be noted that we use generators for generating meaningful input to the programs, to be able to test with all kinds of valid input.

% comparing running programs through evaluator and vc generator
The second type of test asserts that provably correct programs will evaluate correctly as well.
This is tested by first feeding the programs to the VC generator coupled with Z3, and then to the \textit{Interpreter} with random input.
Given that the program is provable the \textit{Interpreter} evaluates all assertions to \textit{true}.
Note that if we only show partial correctness, the \textit{Interpreter} might run forever, so we only test for terminating instances.
%If the program is falsified, then the evaluator should fail when given the input values from the counter example.
If the program is falsified, then the test will run the program with the falsifiable instance and assert that the \textit{Interpreter} will terminate abnormally.
% Ideally we would extract the counter example from the prover, and assert that the evaluator fails with the given input values. If this does not hold, the assertions are not sufficient to prove correctness of the given program.
% However, right now the test implements the simple approach, and does not extract the counter example.
% This property is also addressed in the next paragraph, where we describe why this is only true in one direction.
% TODO: har vi overhoved gjort det her??

\paragraph{Provable by VC generation ensures successful evaluation.}
% Provable => true in evaluation, but not necessarily the other way around
% Here we want to address how the fact that a program is provable with c generation and SMT solving means that the program will also evaluate corretly, but that it is not necessarily the case the other way around.

As described, we value consistency highly, and it should always hold true that if we can prove total correctness of a program, then the dynamic evaluation should give the expected result. Once again it is important to note this will only hold for total correctness and not necessarily for partial correctness.
% As described above, an important property on the relation between the dynamic execution of a program through the evaluator, and the static proof of said program, is that if we can correctly prove the correctness of a program, then the dynamic evaluation should also hold.
Oppositely, correct dynamic evaluation does not imply provable correctness.
The reason for this is that we might not have provided strong enough assertions to generate an appropriate verification condition, whilst the dynamic execution just needs all assertions to evaluate to \textit{true}. 
This might be even more apparent, considering that quantifiers does not work correctly in the \textit{Interpreter}.
If for example we have a program that uses \textit{true} as a loop invariant, this will hold in each iteration during the dynamic execution, but will probably not be enough to prove any postcondition statically.

Lets take a closer look at the multiplication example program from \cref{figure:mult}. 
% \lstinputlisting{Examplecode/mult.ifc}
% \begin{lstlisting}
% vars: [q,r]
% requirements: {q >= 0 /\ r >= 0}
% <!=_=!>
% res := 0;
% $a := q;
% while (q > 0) ?{res = ($a - q) * r /\ q >= 0} !{q} {
%       res := res + r;
%       q := q - 1;
% };
% #{res = $a * r};
% \end{lstlisting}
We have previously argued that the code is correct and can correctly be proved by Z3, but if we relax some of the assertions in the program, this will no longer be the case.
Consider exchanging the loop-invariant \verb|?{res = ($a - q) * r /\ q >= 0}| with the looser invariant \verb\?{res = ($a - q) * r}\.
Now Z3 will no longer be able to prove the correctness of the program. The generated formula looks as shown in \cref{fig:modmult}.

\begin{figure}
\begin{lstlisting}[mathescape=true]
$\forall q, r. \; (q \geq 0 \land r \geq 0) \Rightarrow$
$ \quad \forall res_{3}. \; res_{3} = 0 \Rightarrow$
$\quad \quad \forall \$a. \; \$a = q \Rightarrow$
$\quad \quad \quad (res_{3} = (\$a - q) * r \land q \geq 0)$
$\quad \quad \quad \; \land \forall q_{2}, res_{2}, \xi_{1}.$
$\quad \quad \quad \quad (q_{2} > 0 \land res_{2} = (\$a - q_{2}) * r \land \xi_{1} = q_{2}) \Rightarrow$
$\quad \quad \quad \quad \quad \forall res_{1}. \; res_{1} = res_{2} + r \Rightarrow$
$\quad \quad \quad \quad \quad \quad \forall q_{1}. \, q_{1} = q_{2} - 1 \Rightarrow$
$\quad \quad \quad \quad \quad \quad \quad (res_{1} = (\$a - q_{1}) * r \land 0 \le \xi_{1} \land q_{1} < \xi_{1})$
$\quad \quad \quad \land ((q_{2} \leq 0 \land res_{2} = (\$a - q_{2}) * r) \Rightarrow res_{2} = \$a * r)$
\end{lstlisting}
\caption{Generated verification condition for the modified \texttt{mult} program.}
\label{fig:modmult}
\end{figure}

It becomes quite apparent that the invariant is no longer strong enough to prove the condition, since the restriction on $q_{2}$ is too weak. 
Z3 gives us a falsifiable example where $q_2 = -3$, $res_2 = 6$, $\$a = 0$ and $r=2$.
Using these values, the two first conjunctions in line $4$ and lines $5$-$9$ will evaluate to \textit{true}.
In the last term in line $10$, the LHS of the implication will evaluate to \textit{true} as
$(-3 \leq 0 \land 6 = 3 * 2)$
is correct, but the RHS will evaluate to \textit{false}, since
$6 \neq 0 * 2$
is not correct. Thus the counter example does in fact falsify the verification condition.

But we know from testing that the program does in fact behave as intended, so how can the prover falsifies the formula? 
We can verify that our application acts correctly, by doing the same modification to the whyML program in \cref{fig:why3}.
Trying to prove the program correctness gives a falsifiable counter example, as expected.
By this, we have confidence that our implementation works correctly, and requires the necessary loop invariants.
% \\~\\




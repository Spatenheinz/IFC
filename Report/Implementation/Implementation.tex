\section{Implementation}
IFC is yet to be more than just a toy-language, however in the design of the language and the actual compiler, we have tried to focus making the code modular and easy to extend.
Currently the program consists of 4 parts.
\begin{enumerate}
  \item Parser
  \item Evaluator
  \item Verification Condition Generator
  \item External SMT-solver API
\end{enumerate}
Each part will carry out a task for the program. There is no explicit dependency between any of these parts. Though the main interface is setup to the following tasks:
\begin{enumerate}
  \item Extract the Abstract Syntax Tree generated in the Parser
  \item Run a program with an initial store (TODO!!!! REALLY NEED THIS)
  \item Extract the formula by Verification Condition Generator
  \item Run the formular through an external prover (Z3).
\end{enumerate}
Section ?? explains each of these are used. In the following section we describe how each the 4 program parts are implemented.

\subsection{Parser}\label{sec:parser}
The parser parses an input program, written in \textit{While}, into an abstract syntax tree.
The abstract syntax is very similar to the grammar in \cref{table:grammar} and can be seen in \cref{sec:ast}. 
% Two matters are important to state about the parsing stage, ambiguity and ghost-variables.
Most of the implementation follows directly from the grammar, however, in the matter of resolving ambiguity, and introducing of ghost variables, we have made non-trivial design choices.
These are presented in the following subsections.

\subsubsection{Grammar ambiguity}
The parser is built using the parser combinator library MegaParsec, which means we have to eliminate ambiguity in the grammar presented.
We do so in two different ways.
For arithmetic operators and boolean operators, we make use of the expression parser defined in \verb\Control.Monad.Combinators.Expr\.
This makes handling operators, precedence and associativity easy. Furthermore, it allows for easy extension with new operators.

For parsing first order logic in assertions, we found the expression-parser unfit. 
One reason for this is that we want to allow for syntactic sugar, such as ``$\forall$ x y z.'', which should desugar to ``$\forall$ x. $\forall$ y. $\forall$ z''. 
Therefore, instead of using the expression-parser, we manually introduce precedence, according to conventions.
That is, negation binds tightest, then conjuction, disjuction, quantifers, and lastly implication. 
Furthermore the grammar has been left-factorized. The resulting grammar can be seen in \cref{fig:parsergrammar}.

\begin{figure}[h!]
\begin{grammar}
<quant> ::= 'forall' <vname> '.' <quant>
\alt 'exists' <vname> '.' <quant>
\alt <imp>

<imp> ::= <cd> <imp'>

<imp'> ::= '$\Rightarrow$' <cd> <quant> | $\epsilon$

<cd> ::= <neg> <cd'>

<cd'> ::= '$\wedge$' <neg> <cd'> | '$\vee$' <neg> <cd'> | $\epsilon$

<neg> ::= '$\sim$' <factor> | <factor>

<factor> ::=  <bexp> | '(' <quant> ')'
\end{grammar}

\caption{Modified grammar of \textit{While}.}
\label{fig:parsergrammar}
\end{figure}

We also introduce some syntactic sugar in the grammar, such as ``if c \{s\};'' which will be desugared into ``if c \{s\} else \{skip\};''. 
This can easily be resolved in the parser by using the  \verb\option\ parser-combinator.
\\~\\
For other parts of the grammar, which could have been syntactic sugar, such as implication and exist, we use the unsugared constructs. Although this introduces more code, the intent is to make the code easier to reason about in terms of the semantics. Furthermore it reads better in the output of the vc generator, and hence translate more directly into the predicate transformer semantics.
This at least has made the development process easier. Ideally this could be alleviated by a more comprehensible pretty-printer, but at the moment, we settle for a slightly bigger AST.

\subsubsection{Ghost variables}
As previously mentioned in \cref{sec:hoare}, it is semantically disallowed to use ghost variables anywhere in the program logic, except in assertions. 
Although this is a semantical matter, we handle this in the parser. 
If a ghost variable occurs in an invalid context, parsing simply fails, thus treating it as a syntactic issue.
We do so by adding a reader monad transformer to the internal transformer type \verb\parsecT\ of \verb\Megaparsec\.

\begin{lstlisting}{haskell}
type Parser = ParsecT Void String (ReaderT Bool Identity)
\end{lstlisting}

By changing the boolean value in this environment when entering an assertion context, we denote whether we are allowed to parse ghost variables at a certain point in the program.
By only setting the boolean to \textit{true} when in an assertion environment, we ensure that parsing of ghost variables will never happen in the program logic.
%The boolean value in this environment will tell if the next parser (by the use of \verb\local\) must allow for parsing ghost variables. Which will certainly only happen in assertions.
% We find that eliminating illegal usecases for ghosts in the parser is far preferable than doing so in both the VC-generator and the evaluator, however this restricts us from generating ghost variables in our Quickcheck generation of statements.
% Having a local environment denoting if something is legal or not showed to be very useful, when handling assertions in the evaluator. An explanation of this can be seen in \cref{sec:evaluator}.


\subsection{Evaluator} \label{sec:evaluator}
The evaluator follows directly from the semantics presented in \autoref{sec:Language}.
That is the intial store provided to the evalutor will be modified over the course of the program according to the semantics.
We define the \verb\Eval\ type as follows such:

\begin{lstlisting}
type STEnv = M.Map VName Integer
type Eval a = RWST () () STEnv (Either String) a
\end{lstlisting}

At the moment there is no use for reader, however when the language in the future is extended to have procedures the reader monad will be a natural choice for the scoping rules of said procedures.
Likewise there is no real use of the writer monad yet. Again this is for proofing for a future where the language supports some sort of IO, and atleast being able to print would be nice.
The store described in \autoref{sec:Language} is kept in the State.
The Store is simply a map from VName to Integers. We want the store to be a State as the store after one monadic action should be chained with the next monadic action. This ensures that all variables are in scope for the rest of the program and easily allows for mutable variables.
Duely note that ghost variables will also reside in this environment but will not be mutable or even callable as previously explained.
We make use of the error monad to resolve any runtime-errors that would arise, that is if a violation is done, a ghost is assigned, a variable is used before it is defined or if undefined behaviour arises such as division by 0.
\\~\\
Each non-terminal in the grammar are resolved by different functions which all operate under the $Eval$ monad, which allows for a clean and modular monadic compiler.
\\~\\
One important note about the interpreter is that we have no good way of checking assertions which includes quantifiers. The reasons is that we wanted (possibly erronously) to support arbitrary precision integers.
This entails that we currently has no feasible way to check such assertions.
A potential solution would be to generate a symbolic reprensentation of the formula and try to satisfy it by using an external prover.
Though the interpreter would then also require external dependencies, and not be a standalone program any longer.
All in all, this is unideal, and currently the approach is to ``ignore'' such assertions by considering them true. In \autoref{sec:typesystem} we describe a potential extension to IFC, which could help alleviate this problem.
Non-qunatified assertions, will still be evauluated as per the operational semantics.


\subsection{Verification Condition Generator}
The verification condition generator, uses the weakest precondition calculus to construct the condition, or weakest liberal precondition (if while has no specified variant). Like the approach in the evaluator we want to be able to chain the actions and include a state and reader environment in the construction of the verification condition.

\begin{lstlisting}
type Counter = M.Map VName Count
type Env = M.Map VName VName
type WP a = StateT Counter (ReaderT Env (Either String)) a
\end{lstlisting}

The Counter state is used to give unique names to variables. The purpose of the reader environment is to resolve variable substitution in the formula generated from the weakest preconditions.
\\~\\
As described in \cref{sec:vcg}, we use the quantified rule for assignment, when encountering assignments, to easily handle multiple occurences of the same variable. When developing the code for the VC generator our initial approach tried to minimize the number of times we had to resolve condition $Q$, but our approach to this showed failed. The working solution on the other hand is naive.

\subsubsection{Failed attempt}\footnote{should this even be included?}
As mentioned we tried to minimize the number of times we had to traverse condition $Q$. The initial approach tried to resolve $Q$ only after the entire formula were build. This would allows for a sentence with all names resolved in only two passes, one over the imperative language AST and one over the structure of the formula. The approach was intended to build up a map as such, where \verb\VName\ is an identifier:

\begin{lstlisting}
type Env1 = M.Map VName [VName]
\end{lstlisting}

Whenever encountering a variable we would add a unique identifier to its value-list along with extending $Q$ by the WP rules, as such:

\begin{lstlisting}
missing
\end{lstlisting}

The result of running WP would then give a partially resolved formula, meaning that all the bound variables introduced in the quantifiers of the assignment rule will be correctly resolved. All other variables, would have their original name, and hence at this point be free.

The second traversal will the be on this partly resolved formula. Whenever encountering a variable $x$, we would then look it up in the Map, and then replace $x$ with the head of the list, since this must have been the last introduced bound variable. When encountering a quantifier (except for the first for each free variable), we would then ``pop'' the head of the appropriate list, as any later occurence of $x$ would be substituted with this newly found variable. The problem with this approach is how the AST for First Order Logic formula is constructed. Consider the following example:
\begin{lstlisting}
r := 5
r := r + 10
\end{lstlisting}
which after popping $r_{1}$ of the list would then look like
\begin{lstlisting}[mathescape=true]
$\forall r_{1}\; . \; r_{1} = 5 => \forall r_{2} \; . \; r_{2} = r + 10 => true$
\end{lstlisting}
This means that we can no longer substitute in $r_{1}$ when encountering the next $r$ in $r_{2} = r + 10$. On the other hand if we dont ``pop'' as soon as $r_{2}$ is encountered we have no information on when to do so, as it is not necessary the case that there appears an $r$ in the equality before the implication. When this problem arose we turned to the naive solution.

\subsubsection{Successful attempt}
The second and current approach is to resolve $Q$ whenever we encounters an assignment. We generate the forall as such:
\begin{lstlisting}
wp (Assign x a) q = do
    x' <- genVar x
    q' <- local (M.insert x x') $ resolveQ1 q
    return $ Forall x' (Cond (RBinary Eq (Var x') a) .=>. q')
\end{lstlisting}
We make a new variable (by generating a unique identifier, based on the State), then we proceed to resolve $q$ with the new environment, such that every occurence of variable $x$ will be substituted by the newly generated variable $x'$.
The \verb\Aexpr\ which $x$ evaluates to should not be resolved yet, as this will potentially depend on variables not yet encounted.
Ghost variables on the other hand are easy to resolve as they need not be substituted, because of immutability.
\\~\\
The current version does not enforce the formula to be closed, although it will be necessary in the generation of symbolic variables. Although easily fixed by a simple new iteration over the AST of the formula, and checking if any non-ghost variable does not contain a \verb\#\, we find that since the formula is intended to be fed to the next stage in the compiler it is uneccesary to do so.

\subsubsection{While - invariants and variants}
The \texttt{while} statement has the most complicated Weakest Precondition. The code for computing wp for a \texttt{while}-loop is presented in \cref{fig:wpwhile}.
\begin{figure}[h]
\begin{lstlisting}
wp (While b inv var s) q = do
  st <- get
  (fa, var', veq) <- maybe (return (id,
                            Cond $ BoolConst True,
                            Cond $ BoolConst True)) resolveVar var
  w <- wp s (inv ./\. var')
  fas <- findVars s []
  let inner = fa (((Cond b ./\. inv ./\. veq) .=>. w)
                          ./\. ((anegate (Cond b) ./\. inv) .=>. q))
  env <- ask
  let env' = foldr (\\(x,y) a -> M.insert x y a) env fas
  inner' <- local (const env') $ resolveQ1 inner
  let fas' = foldr (Forall . snd) inner' fas
  return $ inv ./\. fas'
  where
    resolveVar :: Variant -> WP (FOL -> FOL, FOL, FOL)
    resolveVar var = do
      x <- genVar "variant"
      return (Forall x,
              Cond (Negate (RBinary Greater (IntConst 0) (Var x)))
               ./\. Cond (RBinary Less var (Var x))
             , Cond (RBinary Eq (Var x) var)
             )
\end{lstlisting}
\caption{Weakest precondition for \texttt{while}}
\label{fig:wpwhile}
\end{figure}

We have tried to make the code generic in terms existence of the variant to eliminate code duplication.
line 6-9, will generate the conditions needed for the variant.
In case no variant is defined we use that predicate-logic and conjuction forms a monoid with $\top$ as identity element, such that, we generate no new $\forall$-quantification, and have subformular $b \Rightarrow WP(s,b)$.
Is there a variant, we generate the equality $\xi = v$, along with the well-founded relation for unbounded integers. This approach should translate pretty well, with possible other types that have a well-founded relation. The rest of the code simply checks which variables are assigned in the body of the while-loop, and generate a variable for each. Collectively this code will generate the weakest precondition for a while statement as per described in equation~\ref{eq:wpwhile}.


\subsection{Proof-assistant API}\label{sec:api}
The proof-assistant API uses the SMT Based Verification library (SBV), which simplifies symbolic programming in Haskell. 
The library is quite generic and extensive compared to what we need.
We mostly make use of the higher level functions, not utilising any internal functions.

Because the default type of SBV does not quite fit our needs, we instead use the provided transformer \texttt{SymbolicT}, to embed the \texttt{Except} monad. 
We want to do so as, when iterating over the formular, we might encounter a variable not yet defined and here fail gracefully, instead of throwing an error.
Generating a \verb\Predicate ~ Sym SBool\ is relatively simple, since the formula generated in the previous stage is already a first order logic formula, making it straight forward to convert it into SBV's types. 
For the entire highlevel logic we resolve it as presented in \cref{fig:tosym}.
\begin{figure}[h!]
\begin{lstlisting}
type Sym a = SymbolicT (ExceptT String IO) a
type SymTable = M.Map VName SInteger

fToS :: FOL -> SymTable -> Sym SBool
fToS (Cond b) st = bToS b st
fToS (Forall x a) st = forAll [x] $ \(x'::SInteger) ->
  fToS a (M.insert x x' st)
fToS (Exists x a) st = forSome [x] $ \(x'::SInteger) ->
  fToS a (M.insert x x' st)
fToS (ANegate a) st = sNot <$> fToS a st
fToS (AConj a b) st = onlM2 (.&&) (`fToS` st) a b
fToS (ADisj a b) st = onlM2 (.||) (`fToS` st) a b
fToS (AImp a b) st = onlM2 (.=>) (`fToS` st) a b
\end{lstlisting}
\caption{Code for converting a first order logic formular into a symbolic bool.}
\label{fig:tosym}
\end{figure}

It is equally straight forward to resolve \verb\bexpr\ and \verb\aexpr\. 
Ideally we would add a ReaderT to the transformer-stack to get rid of the explicit state.
We have not been able to resolve the type for this, because of the following type constraint 
\begin{verbatim}
forAll :: MProvable m a => [String] -> a -> SymbolicT m SBool
\end{verbatim}
and because \texttt{m} must be an \texttt{ExtractIO}, which Reader and State does not implement, and which by the documentation cannot be implemented.

The predicate constructed by traversing the formula from the last stage is then be used as argument for the SBV function \verb\prove\, which try to prove the predicate using Z3.
If the program can be proved by the external SMT solver, the output will be \texttt{Q.E.D.}.
If the formula is falsifiable, a caounter example is presented.
For instance the output of the following program will obviously always be falsified.
\begin{lstlisting}
violate;
\end{lstlisting}
whereas the multiplication program in \cref{figure:mult} is provable.


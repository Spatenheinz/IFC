\subsection{Interface for proofs}\label{sec:interface}
As described briefly in \cref{sec:Language} we require that programs have a program header.
This can also be seen from the example program presented earlier in this report.
The header must look as follows:
\begin{lstlisting}
vars: [ <variables> ]
requirements: { <preconditions> }
<!=_=!>
\end{lstlisting}
where \<variables\> describes the variables initially in the store, and \<preconditions\> is zero or more assertions, which specifies properties that must hold before program start.
The header does not affect the interpretation of a program as such, besides prepending the requirements as assertions that must hold in the beginning of the program, but it enables us to make more generic proofs about said program.

In the current implementation, there are no procedures in \textit{While}, which makes it difficult to reason about the input of variables.
Therefore we need the header to be able to generate and prove the weakest precondition for a program.
The inspiration comes from how the WhyML language defines procedures.
\cref{fig:why3} is a WhyML program equivalent to the \texttt{mult.ifc} program.

\begin{figure}[h]
\begin{lstlisting}
module Mult

  use int.Int
  use ref.Refint

  let mult (&q : ref int) (r: int) : int
    requires { q >= 0}
    =
    let ref res = 0 in
    let ghost a = q in
    while q > 0 do
      invariant { res = (a - q) * r && q >= 0}
      variant { q }
      decr q;
      res += r
    done;
    assert { res = a * r };
    res
end
\end{lstlisting}
\caption{Why3 program equivalent to mult.ifc}
\label{fig:why3}
\end{figure}

It is possible for why3 to generate a vector of input variables and then a precondition for each the conditions in \texttt{requires}, such that 
$$\forall x_{1},...,x_{n}. \; requires => WP(body, ensures)$$
That is, whenever the \textit{requires} holds, then the weakest precondition of the body should hold, where \textit{ensures} states the postcondition.
Notice that \cref{fig:why3} does not state any \textit{ensures}, although it is usually the idomatic way of writing WhyML code, we ommit it to closely resemble our language.
And since we dont have any return values, we dont really need \textit{ensures}, since this might as well be an assertion in the actual program.
%\footnote{is this clear if you dont know whyML?}

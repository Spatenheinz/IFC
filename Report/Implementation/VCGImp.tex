\subsection{Verification Condition Generator}
The verification condition generator uses weakest precondition calculus, or weakest liberal precondition if a while loop has no specified variant, to construct the verification condition of a program. 
Like the approach in the interpreter we want to chain the actions and include a state and reader environment in the construction of the verification condition.
This is done by using the following code:

\begin{lstlisting}
type Counter = M.Map VName Count
type Env = M.Map VName VName
type WP a = StateT Counter (ReaderT Env (Either String)) a
\end{lstlisting}

The \texttt{Counter} state is used to give unique names to variables. The purpose of the reader environment is to resolve variable substitution in the formula generated from the weakest precondition calculus.
\\~\\
As described in \cref{sec:vcg}, we use the quantified rule for assignment, when encountering assignments, to easily handle multiple occurences of the same variable. 
In our initial approach for implementing the VC generator, we tried to minimize the number of times we had to resolve condition $Q$, but our approach to this did not work. The working solution is a more naive approach. The next subsections presents both the initial attempt and the current approach.

\subsubsection{First attempt}
The initial approach tried to minimize the number of times we traversed condition $Q$.
we tried to only resolve $Q$ once, after the entire formula was built.
This would allow resolving a sentence in just two passes, one over the imperative language AST and one over the structure of the formula. 
The approach was intended to build up a map as such, where \verb\VName\ is an identifier:

\begin{lstlisting}
type Env1 = M.Map VName [VName]
\end{lstlisting}

Whenever encountering a variable we would add a unique identifier to its value-list along and modify $Q$ by the weakest precondition for assignment.
the initial pass over the AST of the program, would then be a partially resolved formula.
The second traversal will be on this partly resolved formula. Whenever encountering a variable $x$, we would then look it up in the Map, and then replace $x$ with the head of the list, since this must have been the last introduced bound variable. The values of the map can be seen as a sort of stacks. To resolve $x$ with the appropriate bound variable, we would update the list each time a quantifier is encountered.
The first time a quantifier is encountered we do nothing, as this simply means the first bound variable in the list is the one that should currently be used. 
Next time we encounter a quantification we would ``pop'' the first element, thus moving on to the next bound variable, and all occurrences of $x$ from that point on will be substituted with the new bound variable, until encountering yet another quantification.

The problem with this approach is how our AST for First Order Logic formula is constructed. Consider the following example:
\begin{lstlisting}
r := 5
r := r + 10
\end{lstlisting}
The first traversal would give the paritally resolved formula
\begin{lstlisting}[mathescape=true]
$\forall r\; . \; r = 5 => \forall r \; . \; r = r + 10 => true$
\end{lstlisting}
and a tuple $(r, [r_1, r_2])$ for holding the list of bound variables for $r$.
Now the result of the second traversal would then look like
\begin{lstlisting}[mathescape=true]
$\forall r_{1}\; . \; r_{1} = 5 => \forall r_{2} \; . \; r_{2} = r_2 + 10 => true$
\end{lstlisting}
As can should be apparent that the last $r$ in the formula should actually be substituted with $r_1$ and not by $r_{2}$ to make the formula correct according to the rules. But as $r_1$ has already been popped off the stack we substitute with $r_{2}$ instead, which is wrong.
% This means that we can no longer substitute in $r_{1}$ when encountering the next $r$ in $r_{2} = r + 10$. 
On the other hand, if we dont ``pop'' as soon as the second quantification is encountered we have no information on when to do so, as it is not necessarily the case that there appears an $r$ in the equality before the implication. 
When this problem was dicovered we turned to the naive solution presented in the next subsection.
It might be possible to include additional pattern matches such as \texttt{Forall x (AImp (Eq (Var y) r) rest)}, where r would then contain $x_0$, where $y_0 = y$. Here $y$ should be substituted by x whereas $y_0$ should be substituted by the recently popped identifier. However we decided to go for the simpler approach.

\subsubsection{Second attempt}
The second and current approach is to resolve $Q$ whenever we encounter an assignment. We generate the forall as such:
\begin{lstlisting}
wp (Assign x a) q = do
    x' <- genVar x
    q' <- local (M.insert x x') $ resolveQ1 q
    return $ Forall x' (Cond (RBinary Eq (Var x') a) .=>. q')
\end{lstlisting}
First we make a new variable by generating a unique identifier, based on the State. 
Then we proceed to resolve $q$ with the new environment, such that every occurence of variable $x$ will be substituted by the newly generated variable stored in $x'$.
The \verb\Aexpr\ which $x$ evaluates to should not be resolved yet, as it could depend on variables not yet encounted.
Ghost variables are easy to resolve, as they are immutable, and thus can be resolved right away.
\\~\\
The current version does not require the formula to be closed, but this is necessary for generating symbolic variables. 
To resolve this, we could simply do another iteration over the formula, checking that all program variables contain a \verb\#\.
However, we find that since the formula is intended to be fed to the next stage in the compiler, it is uneccesary to do so.

\subsubsection{While loops with invariants and variants}
The \texttt{while} statement has the most complicated rule in the weakest precondition calculus. 
The code for computing wp for a while loop is presented in \cref{fig:wpwhile}.
\begin{figure}[h]
\begin{lstlisting}
wlp (While b inv var s) q = do
  (fa, var', veq) <- maybe (return (id, ftrue, ftrue)) resolveVar var
  inv' <- fixFOL inv
  bs' <- fixBExpr b
  var'' <- maybe (return []) fixAExpr var
  w <- wlp s (inv ./\. var')
  fas <- findVars s []
  let inner = fa \$ reqs (((Cond b ./\. inv ./\. veq) .=>. w)
                          ./\. ((anegate (Cond b) ./\. inv) .=>. q)) (inv' <> bs' <> var'')
  env <- ask
  let env' = foldr (\(x,y) a -> M.insert x y a) env fas
  inner' <- local (const env') \$ resolveQ inner
  let fas' = foldr (Forall . snd) inner' fas
  return \$ inv ./\. fas'
  where
    resolveVar :: Variant -> WP (FOL -> FOL, FOL, FOL)
    resolveVar v = do
      x <- genVar "variant"
      let cs' = Cond (bnegate (RBinary Greater (IntConst 0) (Var x))) ./\.
               Cond (RBinary Less v (Var x))
      return (Forall x, cs', Cond (RBinary Eq (Var x) v))
\end{lstlisting}
\caption{Weakest precondition for \texttt{while}}
\label{fig:wpwhile}
\end{figure}

We attempt to make the code generic in terms existence of the variant, to eliminate code duplication.
\texttt{ResolveVar} generate the conditions needed for the variant.
If no variant is defined we produce a triple of identities.
Is there a variant, we generate the equality $\xi = v$, along with the well-founded relation for unbounded integers.
In the future we might have multiple ways of defining a well-founded relations, but this approach should translate well to other types.
Line 3-5 will check if we need conditions for Modulo and Division by 0, of which the approach is presented in \cref{sec:undef}. The rest of the code simply checks which variables are assigned in the body of the while loop, and generates a variable for each. And constructs the weakest precondition as per \cref{fig:wlp} or \cref{fig:wp}, depending on the presence of an invariant. Here it is important to note that the tool we provide does not itself distiguish between partial correctness and total correctness, this responsibility lies solely at the user.

\subsubsection{Handling undefined behaviour in Verification Condition Generator}\label{sec:undef}

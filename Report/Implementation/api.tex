\subsection{Proof-assistant API}\label{sec:api}
The proof-assistant API uses the SMT Based Verification library (SBV), which simplifies symbolic programming in Haskell. 
The library is quite generic and extensive compared to what we need.
We mostly make use of the higher level functions, not utilising any internal functions.

Because the default type of SBV does not quite fit our needs, we instead use the provided transformer \texttt{SymbolicT}, to embed the \texttt{Except} monad. 
We want to do so as, when iterating over the formular, we might encounter a variable not yet defined and here fail gracefully, instead of throwing an error.
Generating a \verb\Predicate ~ Sym SBool\ is relatively simple, since the formula generated in the previous stage is already a first order logic formula, making it straight forward to convert it into SBV's types. 
For the entire highlevel logic we resolve it as presented in \cref{fig:tosym}.
\begin{figure}[h!]
\begin{lstlisting}
type Sym a = SymbolicT (ExceptT String IO) a
type SymTable = M.Map VName SInteger

fToS :: FOL -> SymTable -> Sym SBool
fToS (Cond b) st = bToS b st
fToS (Forall x a) st = forAll [x] $ \(x'::SInteger) ->
  fToS a (M.insert x x' st)
fToS (Exists x a) st = forSome [x] $ \(x'::SInteger) ->
  fToS a (M.insert x x' st)
fToS (ANegate a) st = sNot <$> fToS a st
fToS (AConj a b) st = onlM2 (.&&) (`fToS` st) a b
fToS (ADisj a b) st = onlM2 (.||) (`fToS` st) a b
fToS (AImp a b) st = onlM2 (.=>) (`fToS` st) a b
\end{lstlisting}
\caption{Code for converting a first order logic formular into a symbolic bool.}
\label{fig:tosym}
\end{figure}

It is equally straight forward to resolve \verb\bexpr\ and \verb\aexpr\. 
Ideally we would add a ReaderT to the transformer-stack to get rid of the explicit state.
We have not been able to resolve the type for this, because of the following type constraint 
\begin{verbatim}
forAll :: MProvable m a => [String] -> a -> SymbolicT m SBool
\end{verbatim}
and because \texttt{m} must be an \texttt{ExtractIO}, which Reader and State does not implement, and which by the documentation cannot be implemented.

The predicate constructed by traversing the formula from the last stage is then be used as argument for the SBV function \verb\prove\, which try to prove the predicate using Z3.
If the program can be proved by the external SMT solver, the output will be \texttt{Q.E.D.}.
If the formula is falsifiable, a caounter example is presented.
For instance the output of the following program will obviously always be falsified.
\begin{lstlisting}
violate;
\end{lstlisting}
whereas the multiplication program in \cref{figure:mult} is provable.

%- snak om Weakest precondition, i en figur?
% the loosest precondition. Q is wp if {P}S{R} if and only if P=>Q. The least requirement for guaranteeing that R holds after S.
% There is only one wp. If Q and Q' are both wp, then {Q'}S{R} iff Q'=> Q and {Q}S{R} iff Q=>Q', so Q=Q', thus the wp is unique. (wiki)
%			wp vs sp? : {P}S{R} iff slp P c => R, meaning is is the strongest postcondition, ie if Q is sp then {P}S{R} if and only if Q => R. (2.pdf)

% verification conditions

% - Beskrive de vigtige af dem?
% assign, seq, while
% wlp vs wp: partial correctness by ignoring termination. (1) invariant must hold at the start of the loop, (2) the invariant and the loop condition is enough to find the weakest precondition necessary to re-establish the loop invariant, (3) if and when the loop terminates using the invariant and the fact that the loop condition is false is enough to establish the required postcondition (wiki)

Below the structure for computing the weakest liberal precondition for the different constructs is shown.
\begin{align*}
	WLP(\texttt{skip}, Q) &= Q \\
	WLP(x:=e,Q) &= Q[x \leftarrow e] \\
	WLP(s_1;s_2, Q) &= WLP(s_1, WLP(s_2, Q)) \\
	WLP(\texttt{if } e \texttt{ then } s_1 \texttt{ else } s_2, Q) &= (e \neq 0 \Rightarrow WLP(s_1, Q)) \land (e = 0 \Rightarrow WLP(s_2, Q)) \\
	WLP(\texttt{while } e \texttt{ invariant } I \texttt{ do } s, Q) &= 
		I \land \\
    &\quad \forall x_1, ..., x_k, \\
    &\quad (((e \neq 0 \land I) \Rightarrow WLP(s, I)) \land ((e = 0 \land I) \Rightarrow Q))
       [w_i \leftarrow x_i] \\
	&\quad \text{where } w_1, ..., w_k \text{ is the set of assigned variables in} \\
  &\quad \text{statement } s \text{ and } x_1, ..., x_k \text{ are fresh logical variables.}
\end{align*}


% hvad med for total correctness?
% to prove termination of while loop we need variant

Modified syntax for \textit{while}-loop, in order to prove termination using weakest precondition.
$$
	\frac{
		\{I \land e \neq 0 \land v = \xi \} s \{I \land v \prec \xi \} \quad wf(\prec)
	}{
		\{I\} \texttt{ while } e \texttt{ invariant } I 
          \texttt{ variant } v, \prec \texttt{ do } s \{I \land e = 0\}
	}
$$

Now the weakest liberal precondition does not prove termination. If we want to prove termination in addition to the partial correctness obtained from \textit{wlp}, we need a \textit{weakest precondition} which is much like \textit{wlp}, but require that \textit{while}-loops have a loop variant. This makes the difference between partial and total correctness.

The loop variant is an expression that decreases, for example in the \textit{while}-loop form our example program decreases $q$ in each iteration, as can be seen in line 8 of the code (see code listing \autoref{figure:mult}).

Now the structure for computing the weakest preconditions for the constructs for total correctness is much like the one for computing weakest liberal precondition, except for the structure of \textit{while}-loops, which can be seen below.

\begin{align*}
	WP\left(
     \begin{array}{c}
     \texttt{while } e \texttt{ invariant } I \\
     \texttt{ variant } v, \prec \texttt{ do } s
    \end{array}
    , Q \right) 
    &= 
		I \land \\
    &\quad \forall x_1, ..., x_k, \xi, \\
    &\quad \quad (((e \neq 0 \land I \land \xi = v) \Rightarrow WP(s, I \land v \prec \xi)) \\
    &\quad \quad \quad \land ((e = 0 \land I) \Rightarrow Q)) [w_i \leftarrow x_i] \\
	&\quad \text{where } w_1, ..., w_k \text{ is the set of assigned variables in} \\
  &\quad \text{statement } s \text{ and } x_1, ..., x_k, \xi \text{ are fresh logical variables.}
\end{align*}

%- gennemgå det i eksemplet
% hvert fald for noget af eksemplet ?

% Sound og Complete?
%   - Soundness:
%				If {P}S{Q} can be derived, then {{P}}S{{Q}} holds
%				Any derivable triple is valid
%   - Completeness:
%       If {{P}}S{{Q}} holds, then {P}S{Q} can be derived
%				If the language is expressive enough, any valid triple {P}s{Q} can be derived



% verification conditions
In the previous section, we presented Hoare Triples and how they can be used to assign meaning to programs. 
As mentioned, for the postcondition to hold true, we want the program to be executed in a state satisfying the precondition. 
It is fair to assume that we know the postcondition of a program, but how can we know that our precondition is saying anything meaningful about our program? 
We can use \textit{Weakest Precondition Calculus} (denoted $WP$) to generate a suitable precondition in such a case.
It essentially states:
$$
\{WP(S,Q)\}S\{Q\}
$$
WP is a well-defined calculus to find the least restrictive precondition that will make $Q$ hold after $S$. 
By this calculus, which we will present in this section, validation of Hoare Triples can be reduced to a single logical sentence. WP is functional compared to the relational nature of Hoare logic, which means that is easier to reason about, as we dont have to consider any consequences.
By computing the weakest precondition, we get a formula of first-order logic that can be used to verify the program, and that process is called \textit{verification condition generation}.

\subsubsection{Weakest Liberal Precondition}
The \textit{Weakest Liberal Precondition} is the weakest precondition necessary for proving correctnes of a program, without proving termination. We denote it by $WLP$.
In \cref{fig:wlp}, the structure for computing the WLP for the different constructs of \textit{While} is shown.

\begin{figure}
\begin{align*}
WLP(\texttt{violate}, Q) &= false \\
WLP(\texttt{skip}, Q) &= Q \\
WLP(x:=e,Q) &= \forall y, y = e \Rightarrow Q[x \leftarrow y] \\
WLP(s_1;s_2, Q) &= WLP(s_1, WLP(s_2, Q)) \\
WLP(\texttt{if } e \texttt{ then } s_1 \texttt{ else } s_2, Q) &= (e \Rightarrow WLP(s_1, Q)) \\
    &\quad \land (\neg e \Rightarrow WLP(s_2, Q)) \\
WLP(\texttt{while } e \texttt{ invariant } I \texttt{ do } s, Q) &= 
    I \land \\
&\quad \forall x_1, ..., x_k, \\
&\quad (((e \land I) \Rightarrow WLP(s, I)) \\
&\quad \quad \land (( \neg e \land I) \Rightarrow Q))
    [w_i \leftarrow x_i] \\
&\quad \text{where } w_1, ..., w_k \text{ is the set of} \\
&\quad \text{assigned variables in statement } s \text{ and } \\
&\quad x_1, ..., x_k \text{ are fresh logical variables.} \\
WLP(\{P\}, Q) &= P \land Q \quad \text{where P is an assertion}
\end{align*}
\caption{Rules for computing weakest liberal precondition of a statement in \textit{While}\cite{wlp}.}
\label{fig:wlp}
\end{figure}

Most of the rules are somewhat self-explanatory, but we would like to go through a selection of the rules that we find to be complex in our language setting.

\paragraph{Assignments.}
The rule for computing the WLP for assignments says that for all variables $y$ where $y = e$, we should exchange $x$ in $Q$ with $y$. So we exchange all occurrences of $x$ with the value assigned to $x$.
We use a quantifier for $y$ to avoid exponential growth by letting $y=e$ and substituting $x$ with $y$ in $Q$.
For instance, if $e=1+2+...+100.000$, then by letting $y$ hold the value of $e$, we avoid substituting in this very long expression, making the formula significantly more concise.

\paragraph{Sequence.}
For finding the \textit{wlp} of a sequence of statements $s_1;s_2$, we need to first find the \textit{wlp} $Q'$ of $s_2$ with $Q$, and then compute the \textit{wlp} of $s_1$ with $Q'$.
This shows how we compute the weakest liberal precondition using a bottom-up approach.

\paragraph{While-loops.}
To compute the WLP of a while loop, we have to ensure that the loop invariant holds before, inside, and after the loop.
The first condition is simply that the invariant $I$ must evaluate to $true$ before the loop.
Next, we need to assert that no matter what values the variables inside the loop have, the invariant and loop condition will hold whenever we go into the loop.
Lastly, the invariant and the negated loop condition must hold when the loop terminates.
We must also exchange all the occurrences in the currently accumulated weakest liberal precondition $Q$ for these variables.

\paragraph{Additions to the standard rules.}
Most of the rules are the standard rules for computing the weakest liberal precondition.
We have added two rules: the rule for \texttt{violate}, which will always give \textit{false}, and the rule for assertions, which adds the assertion $P$ to the accumulated condition $Q$, s.t. the new WLP is $P \land Q$.

\paragraph{Resolving undefined behaviour.}
As mentioned in the previous subsection, division and modulo by zero is undefined behavior.
It is resolved by adding a condition whenever encountering a division or modulo operator, verifying that the operation is legal.
For example, if we have an assignment $x = a \% b$, we will need the following precondition
\begin{align*}
\forall y . ( (&\neg (b = 0)) \\
                &\land (b \neq 0 \Rightarrow y = a \% b \Rightarrow Q[x \leftarrow y]))
\end{align*}
which only resolves the assignment if $b$ is not zero, and otherwise will always fail. Resolving division is similar.
This modification is not explicitly written in the rules, but follows from the semantics of our \textit{While} language.

\subsubsection{Weakest Precondition}
The weakest liberal precondition does not prove termination. However, by the variant in the while-loop introduced in \cref{sec:total-correctness}, we can extend the \textit{weakest liberal precondition} to \textit{weakest precondition}, which will also ensure termination.

The structure for computing the WP for the constructs of the language is much like the one for computing the WLP, except for the structure of while loops, which is presented in \cref{fig:wp}.

\begin{figure}[h!]
\begin{align*}
WP\left(
    \begin{array}{c}
    \texttt{while } e \texttt{ invariant } I \\
    \texttt{ variant } v, \prec \texttt{ do } s
    \end{array}
, Q \right) 
&= 
    I \land \label{eq:wpwhile} \\
&\quad \forall x_1, ..., x_k, \xi, \\
&\quad \quad (((e \land I \land \xi = v) \Rightarrow WP(s, I \land v \prec \xi)) \\
&\quad \quad \quad \land ((\neg e \land I) \Rightarrow Q)) [w_i \leftarrow x_i] \\
&\quad \text{where } w_1, ..., w_k \text{ is the set of assigned} \\
&\quad \text{variables in statement } s \text{ and } x_1, ..., x_k, \xi \\
&\quad \text{are fresh logical variables.}
\end{align*}
\caption{The rule for computing the weakest precondition for while loops.}
\label{fig:wp}
\end{figure}

The difference between the computation of WP and WLP for while loops is the presence of the \textit{variant}.
When given a variant $v$ for the loop, we assert that $v$ decreases with each iteration, using the logical variable $\xi$ to keep the old value of $v$ to compare with.

For our example program \texttt{mult.ifc} we can compute the WP, as we have both a variant and an invariant in the while loop.
By applying the WP rules on the example, we get the WP seen in \cref{figure:wpmult}.

\begin{figure}[h!]

% \begin{verbatim}
% Forall q#3 . q#3 = 10 =>
%     Forall r#1 . r#1 = 55 =>
%         (q#3 >= 0 /\ r#1 >= 0) /\ Forall res#3 . res#3 = 0 =>
%             Forall ghosta . ghosta = q#3 =>
%                 res#3 = (ghosta - q#3) * r#1 /\ Forall q#1 . Forall res#1 .
%                     (q#1 >= 0 /\ res#1 = (ghosta - q#1) * r#1) =>
%                         Forall res#2 . res#2 = res#1 + r#1 =>
%                             Forall q#2 . q#2 = q#1 - 1 =>
%                               res#2 = (ghosta - q#2) * r#1
%                         /\ (q#1 <= 0 /\ res#1 = (ghosta - q#1) * r#1) =>
%                               res#1 = ghosta * r#1
% \end{verbatim}

% \begin{align*}
%          &\forall q\#3 . q\#3 = 10 \Rightarrow \\
%          &\quad \forall r\#1 . r\#1 = 55 \Rightarrow \\
%          &\quad \quad (q\#3 \geq 0 \land r\#1 \geq 0)
%          							\land \forall res\#3 . res\#3 = 0 \Rightarrow \\
%          &\quad \quad \quad \quad \forall ghosta . ghosta = q\#3 \Rightarrow \\
%          &\quad \quad \quad \quad \quad res\#3 = (ghosta - q\#3) \cdot r\#1 \\
%          &\quad \quad \quad \quad \quad \land
%            \forall q\#1 . \forall res\#1 .
%               (q\#1 \geq 0 \land res\#1 = (ghosta - q\#1) \cdot r\#1) \Rightarrow \\
%          &\quad \quad \quad \quad \quad \quad
%            \forall res\#2 . res\#2 = res\#1 + r\#1 \Righarrow \\
%          &\quad \quad \quad \quad \quad \quad \quad
%            \forall q\#2 . q\#2 = q\#1 - 1 \Rightarrow \\
%          &\quad \quad \quad \quad \quad \quad \quad \quad
%            res\#2 = (ghosta - q\#2) \cdot r\#1 \\
%          &\quad \quad \quad \quad \quad \quad \land
%            (q\#1 \leq 0 \land res\#1 = (ghosta - q\#1) \cdot r\#1) \Rightarrow \\
%          &\quad \quad \quad \quad \quad \quad \quad
%            res\#1 = ghosta \cdot r\#1
% \end{align*}


% \begin{lstlisting}
% Forall q#3 . q#3 = 10 =>
%     Forall r#1 . r#1 = 55 =>
%         (q#3 >= 0 /\ r#1 >= 0) /\ Forall res#3 . res#3 = 0 =>
%             Forall ghosta . ghosta = q#3 =>
%                 res#3 = (ghosta - q#3) * r#1
%                 /\
%                 Forall q#1 . Forall res#1 .
%                     (q#1 >= 0 /\ res#1 = (ghosta - q#1) * r#1) =>
%                         Forall res#2 . res#2 = res#1 + r#1 =>
%                             Forall q#2 . q#2 = q#1 - 1 =>
%                                 res#2 = (ghosta - q#2) * r#1
%                         /\
%                         (q#1 <= 0 /\ res#1 = (ghosta - q#1) * r#1) =>
%                             res#1 = ghosta * r#1
% \end{lstlisting}

\begin{lstlisting}[mathescape=true]
$\forall \, q_3, \, (q_3 = 10) \Rightarrow$
$\quad \forall r_1 , r_1 = 55 \Rightarrow$
$\quad \quad (q_3 \geq 0 \land r_1 \geq 0)
						\land \forall \, res_3 , \, res_3 = 0 \Rightarrow$
$\quad \quad \quad \quad \forall \, ghosta, \, (ghosta = q_3) \Rightarrow$
$\quad \quad \quad \quad \quad res_3 = (ghosta - q_3) \cdot r_1$
$\quad \quad \quad \quad \quad \land
         \forall q_1 , \forall res_1 , \,
             (q_1 \geq 0 \land res_1 = (ghosta - q_1) \cdot r_1) \Rightarrow$
$ \quad \quad \quad \quad \quad \quad
        \forall res_2 , \, res_2 = (res_1 + r_1) \Rightarrow $
$ \quad \quad \quad \quad \quad \quad \quad
          \forall q_2 , q_2 = (q_1 - 1) \Rightarrow$
$\quad \quad \quad \quad \quad \quad \quad \quad
          res_2 = (ghosta - q_2) \cdot r_1$
$\quad \quad \quad \quad \quad \quad \land
          (q_1 \leq 0 \land res_1 = (ghosta - q_1) \cdot r_1) \Rightarrow$
$\quad \quad \quad \quad \quad \quad \quad res_1 = ghosta \cdot r_1$
\end{lstlisting}

\caption{Code for computing weakest precondition of \texttt{mult.ifc}}
\label{figure:wpmult}
\end{figure}

The interesting thing is how the while loop is resolved. First the \textit{invariant} is checked in line $4$, according to the first part of the \textit{wp} rule.
Next, the rule requires a universal quantifier over all the variables altered in the loop body, which happens in line $5$.
Inside the universal quantifier, we check that the loop invariant and condition hold when entering each iteration (lines $6$-$9$) and that the invariant and the negated loop condition hold when the loop terminates (line $10$).
The variant is used in line $6$, where the logical variable $\xi$ is set to hold the value of the variant, and in line $9$, where it is checked that the variant has decreased by comparing to the initial value of $v$ stored in $\xi$.

%- snak om Weakest precondition, i en figur?
% the loosest precondition. Q is wp if {P}S{R} if and only if P=>Q. The least requirement for guaranteeing that R holds after S.
% There is only one wp. If Q and Q' are both wp, then {Q'}S{R} iff Q'=> Q and {Q}S{R} iff Q=>Q', so Q=Q', thus the wp is unique. (wiki)
%			wp vs sp? : {P}S{R} iff slp P c => R, meaning is is the strongest postcondition, ie if Q is sp then {P}S{R} if and only if Q => R. (2.pdf)

% verification conditions

In the previous section we presented Hoare Triples and how they can be used to ``assign meaning to programs''. However, it might not always be possible that the Precondition is known. In such a case, we can use Weakest Precondition Calculus (denoted $WP$), which essentially states:
$$
 \{WP(S,Q)\}S\{Q\}
$$
That is, we can use a well-defined calculus to find the weakest (least restrictive) precondition that will make $Q$ hold after $S$. By this calculus, which we present in this section, validation of Hoare Triples can be reduced to a logical sentence, since Weakest Precondition calculus is functional compared to the relational nature of Hoare logic.
By computing the weakest precondition, we get a formula of first-order logic that can be used to verify the program, and that is called \textit{verification condition generation}.

\subsubsection{Weakest Liberal Precondition}
Below the structure for computing the weakest liberal precondition for the different constructs is shown.
\begin{align*}
	WLP(\texttt{skip}, Q) &= Q \\
	WLP(x:=e,Q) &= \forall y, y = e \Rightarrow Q[x \leftarrow y] \\
	WLP(s_1;s_2, Q) &= WLP(s_1, WLP(s_2, Q)) \\
	WLP(\texttt{if } e \texttt{ then } s_1 \texttt{ else } s_2, Q) &= (e \Rightarrow WLP(s_1, Q)) \land (\neg e \Rightarrow WLP(s_2, Q)) \\
	WLP(\texttt{while } e \texttt{ invariant } I \texttt{ do } s, Q) &= 
		I \land \\
    &\quad \forall x_1, ..., x_k, \\
    &\quad (((e \land I) \Rightarrow WLP(s, I)) \land (( \neg e \land I) \Rightarrow Q))
       [w_i \leftarrow x_i] \\
	&\quad \text{where } w_1, ..., w_k \text{ is the set of assigned variables in} \\
  &\quad \text{statement } s \text{ and } x_1, ..., x_k \text{ are fresh logical variables.} \\
  WLP(\{P\}, Q) &= P \land Q \quad \text{where P is an assertion}
\end{align*}

% TODO: write something about the above?
The rules are somewhat self-explanatory, but we would like to go through some rules which have been important for our work.

\paragraph{Assignments.}
The rule for computing weakest liberal precondition for assignments says that for all variables $y$ where $y = e$, we should exchange $x$ in $Q$ with $y$. That is, we exchange all occurrences of $x$ with the value that we assign to $x$.

\paragraph{Sequence.}
For finding the \textit{wlp} of a sequence of statements $s_1;s_2$, we need to first find the \textit{wlp} $Q'$ of $s_2$ with $Q$, and then compute the \textit{wlp} of $s_1$ with $Q'$. This shows how we compute the weakest liberal precondition using a bottom-up approach.

\paragraph{While-loops.}
To compute the \textit{wlp} of a while loop we have to eensure that the loop invariant holds before, inside, and after the loop. That is why the first condition is simply that the invariant $I$ must evaluate to $true$.
Next we need to assert that no matter what values the variables inside the loop have, the invariant and loop condition will hold whenever we go into the loop, and the invariant and negated loop condition will hold when the loop terminates.
For these variables, we must also exchange all the occurrences in the currently accumulated weakest liberal precondition $Q$.

\subsubsection{Weakest Precondition}
% hvad med for total correctness?
% to prove termination of while loop we need variant
% wlp vs wp: partial correctness by ignoring termination. (1) invariant must hold at the start of the loop, (2) the invariant and the loop condition is enough to find the weakest precondition necessary to re-establish the loop invariant, (3) if and when the loop terminates using the invariant and the fact that the loop condition is false is enough to establish the required postcondition (wiki)
%Most programs written in our small while-language will trivially terminate, only when using the \textit{while} construct are we concerned with termination. To prove termination of such a loop, one needs one or more loop invariants and possibly a variant.

Now the weakest liberal precondition does not prove termination. If we want to prove termination in addition to the partial correctness obtained from \textit{wlp}, we need a \textit{weakest precondition} which is much like \textit{wlp}, but require that \textit{while}-loops have a loop variant. 
Here we have to use the modified Hoare logic for \textit{while}-loops presented in \ref{sec:total-correctness}.
% This makes it possible to prove termination, and is the difference between partial and total correctness.

The structure for computing the weakest preconditions for the constructs for total correctness is much like the one for computing weakest liberal precondition, except for the structure of \textit{while}-loops, which can be seen below.
\begin{equation}\label{eq:wpwhile}
\begin{align*}
	WP\left(
     \begin{array}{c}
     \texttt{while } e \texttt{ invariant } I \\
     \texttt{ variant } v, \prec \texttt{ do } s
     \end{array}
    , Q \right) 
    &= 
		I \land \label{eq:wpwhile} \\
    &\quad \forall x_1, ..., x_k, \xi, \\
    &\quad \quad (((e \land I \land \xi = v) \Rightarrow WP(s, I \land v \prec \xi)) \\
    &\quad \quad \quad \land ((\neg e \land I) \Rightarrow Q)) [w_i \leftarrow x_i] \\
	&\quad \text{where } w_1, ..., w_k \text{ is the set of assigned variables in} \\
  &\quad \text{statement } s \text{ and } x_1, ..., x_k, \xi \text{ are fresh logical variables.}
\end{align*}
\end{equation}

% What are the differences? explain!
We see that the difference between the computation of \textit{wp} and \textit{wlp} is the presence of the \textit{variant}. When given a variant expression $v$ for the loop, we add the assertion that this expression decreases with each iteration, using the logical variable $\xi$ to keep the old value of $v$ to compare with.

%- gennemgå det i eksemplet
% hvert fald for noget af eksemplet ?
For our example program \texttt{mult.ifc}, we can compute the weakest precondition, as we have both a variant and an invariant in the while loop.
By applying the \textit{wp} rules on the example, we get the weakest precondition seen in \autoref{figure:wpmult}.

\begin{figure}[h]

% \begin{verbatim}
% Forall q#3 . q#3 = 10 =>
%     Forall r#1 . r#1 = 55 =>
%         (q#3 >= 0 /\ r#1 >= 0) /\ Forall res#3 . res#3 = 0 =>
%             Forall ghosta . ghosta = q#3 =>
%                 res#3 = (ghosta - q#3) * r#1 /\ Forall q#1 . Forall res#1 .
%                     (q#1 >= 0 /\ res#1 = (ghosta - q#1) * r#1) =>
%                         Forall res#2 . res#2 = res#1 + r#1 =>
%                             Forall q#2 . q#2 = q#1 - 1 =>
%                               res#2 = (ghosta - q#2) * r#1
%                         /\ (q#1 <= 0 /\ res#1 = (ghosta - q#1) * r#1) =>
%                               res#1 = ghosta * r#1
% \end{verbatim}

% \begin{align*}
%          &\forall q\#3 . q\#3 = 10 \Rightarrow \\
%          &\quad \forall r\#1 . r\#1 = 55 \Rightarrow \\
%          &\quad \quad (q\#3 \geq 0 \land r\#1 \geq 0)
%          							\land \forall res\#3 . res\#3 = 0 \Rightarrow \\
%          &\quad \quad \quad \quad \forall ghosta . ghosta = q\#3 \Rightarrow \\
%          &\quad \quad \quad \quad \quad res\#3 = (ghosta - q\#3) \cdot r\#1 \\
%          &\quad \quad \quad \quad \quad \land
%            \forall q\#1 . \forall res\#1 .
%               (q\#1 \geq 0 \land res\#1 = (ghosta - q\#1) \cdot r\#1) \Rightarrow \\
%          &\quad \quad \quad \quad \quad \quad
%            \forall res\#2 . res\#2 = res\#1 + r\#1 \Righarrow \\
%          &\quad \quad \quad \quad \quad \quad \quad
%            \forall q\#2 . q\#2 = q\#1 - 1 \Rightarrow \\
%          &\quad \quad \quad \quad \quad \quad \quad \quad
%            res\#2 = (ghosta - q\#2) \cdot r\#1 \\
%          &\quad \quad \quad \quad \quad \quad \land
%            (q\#1 \leq 0 \land res\#1 = (ghosta - q\#1) \cdot r\#1) \Rightarrow \\
%          &\quad \quad \quad \quad \quad \quad \quad
%            res\#1 = ghosta \cdot r\#1
% \end{align*}


% \begin{lstlisting}
% Forall q#3 . q#3 = 10 =>
%     Forall r#1 . r#1 = 55 =>
%         (q#3 >= 0 /\ r#1 >= 0) /\ Forall res#3 . res#3 = 0 =>
%             Forall ghosta . ghosta = q#3 =>
%                 res#3 = (ghosta - q#3) * r#1
%                 /\
%                 Forall q#1 . Forall res#1 .
%                     (q#1 >= 0 /\ res#1 = (ghosta - q#1) * r#1) =>
%                         Forall res#2 . res#2 = res#1 + r#1 =>
%                             Forall q#2 . q#2 = q#1 - 1 =>
%                                 res#2 = (ghosta - q#2) * r#1
%                         /\
%                         (q#1 <= 0 /\ res#1 = (ghosta - q#1) * r#1) =>
%                             res#1 = ghosta * r#1
% \end{lstlisting}

\begin{lstlisting}[mathescape=true]
$\forall \, q_3, \, (q_3 = 10) \Rightarrow$
$\quad \forall r_1 , r_1 = 55 \Rightarrow$
$\quad \quad (q_3 \geq 0 \land r_1 \geq 0)
						\land \forall \, res_3 , \, res_3 = 0 \Rightarrow$
$\quad \quad \quad \quad \forall \, ghosta, \, (ghosta = q_3) \Rightarrow$
$\quad \quad \quad \quad \quad res_3 = (ghosta - q_3) \cdot r_1$
$\quad \quad \quad \quad \quad \land
         \forall q_1 , \forall res_1 , \,
             (q_1 \geq 0 \land res_1 = (ghosta - q_1) \cdot r_1) \Rightarrow$
$ \quad \quad \quad \quad \quad \quad
        \forall res_2 , \, res_2 = (res_1 + r_1) \Rightarrow $
$ \quad \quad \quad \quad \quad \quad \quad
          \forall q_2 , q_2 = (q_1 - 1) \Rightarrow$
$\quad \quad \quad \quad \quad \quad \quad \quad
          res_2 = (ghosta - q_2) \cdot r_1$
$\quad \quad \quad \quad \quad \quad \land
          (q_1 \leq 0 \land res_1 = (ghosta - q_1) \cdot r_1) \Rightarrow$
$\quad \quad \quad \quad \quad \quad \quad res_1 = ghosta \cdot r_1$
\end{lstlisting}

\caption{Weakest precondition of \texttt{mult.ifc}}
\label{figure:wpmult}
\end{figure}

The interesting thing is how the \textit{while}-loop is resolved. First the \textit{invariant} is checked in line $4$, according to the first part of the \textit{wp} rule. 
Next the rule requires a forall statement over all the variables altered in the loop bode, and that is what happens in line $5$ of the formula.
Inside this forall we now check that the loop invariant and condition holds when entering each iteration (lines $6$-$9$), and that the invariant and the negated loop condition holds when the loop terminates(line $10$).
The variant is used in line $6$, where the logical variable $\xi$ is set to hold the value of the variant, and in line $9$, where it is checked that the variant has decreased by comparing to the logical variable holding the value that the variant had at the beginning of the loop.


% Sound og Complete?
%   - Soundness:
%				If {P}S{Q} can be derived, then {{P}}S{{Q}} holds
%				Any derivable triple is valid
%   - Completeness:
%       If {{P}}S{{Q}} holds, then {P}S{Q} can be derived
%				If the language is expressive enough, any valid triple {P}s{Q} can be derived



IFC is a small imperative programming language with a build in assertion language enabeling the possibility for statically verifying programs by the use of external provers. In this section we will present the syntax and semantics of both the imperative language and the assertion language.
\begin{table}[h!]
% \centering
\begin{grammar}
<statement> ::= <statement> ';' <statement>
\alt <ghostid> ':=' <aexpr>
\alt <id> ':=' <aexpr>
\alt 'if' <bexpr> '\{' <statement> '\}'
\alt 'if' <bexpr> '\{' <statement> '\}' 'else' '\{' <statement> '\}'
\alt 'while' <bexpr> <invariant> <variant> '\{' <statement> '\}'
\alt '\#\{' <assertion> '\}'
\alt 'skip' | 'violate'

<invariant> ::= '?\{' <assertion> '\}'
\alt <invariant> ';' <invariant>

<variant> ::= '!\{' <aexpr> '\}' | $\epsilon$

<assertion> ::= 'forall' <id> '.' <assertion>
\alt 'exists' <id> '.' <assertion>
\alt '$\sim$' <assertion>
\alt <assertion> <assertionop> <assertion>
\alt <bexpr>

<assertionop> ::= '\verb|/\|' | '\verb|\/|' | '\verb|=>|'

<bexpr> ::= 'true' | 'false' | \alt '!'<bexpr>
\alt <bexpr> <bop> <bexpr>
\alt <bexpr> <rop> <bexpr>
\alt '(' <bexpr> ')'

<bop> ::= '\&\&' | '||' |

<rop> ::= '\verb|<|' | '\verb|<=|' | '\verb|=|' | '\verb|/=|' | '\verb|>|' | '\verb|>=|'

<aexpr> ::= <id>
\alt <ghostid>
\alt <integer>
\alt '-'<axpr>
\alt <aexpr> <aop> <aexpr>
\alt '(' <aexpr> ')'

<aop> ::= '$+$' | '$-$' | '$*$' | '$/$' | '\%'

<ghostid> ::= \emoji{ghost} <string>

<id> ::= <string>
\end{grammar}
\caption{Grammar of IFC}
\label{table:grammar}
\end{table}

\autoref{table:grammar} shows the grammar of IFC. In essence an IFC program is a statement with syntax in the C-family. The language is small, as the focus in this project has been to correctly being able to generate verification conditions for said programs. In Figure ?? we show the semantics of the different statements.

Arithmetic expressions (\verb\aexpr\) follow the standard rules of precedence and associativity:
\begin{itemize}
  \item parenthesis
  \item negation
  \item Multiplication, division and modulo (left associative)
  \item Addition and subtraction (left associative)
\end{itemize}
where parenthesis binds tightest. The semantics is shown in Figure ??.

Likewise boolean expressions


Vi vil gerne give en kort præsentation. og hvorfor det er interessant.
eksempel? mult?
skriv noget om det.

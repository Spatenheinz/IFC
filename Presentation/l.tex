\begin{frame}[containsverbatim]
  \frametitle{Program L}
  $$\vdash \{x \geq 0\} \textbf{ L } \{ true \}$$
\begin{lstlisting}[mathescape=true]
$ y := 0$;
while $x > 1 \lor y > 0$ do
   if $x > 0$ then
      $x := x - 1;$
      $y := y + y;$
    else
      $y := y - 1;$
\end{lstlisting}
\end{frame}

\begin{frame}[containsverbatim]
  \frametitle{Program L}
\begin{lstlisting}[mathescape=true]
vars: [$x$]
requirements: {$x \geq 0$}
<!=_=!>
$y := 1;$
while ($x > 0$ || $y > 0$) ?{$true$} !{$\textcolor{red}{\mathbf{?}}$} {
    if $x > 0$ {
        $x := x - 1;$
        $y := y + y;$
    } else {
        $y := y - 1;$
    };
};
#{$true$};
\end{lstlisting}
Problem is that first $x$ decrements while $y$ increments, then only $y$ decrements.\\
No way to express the variant.
\end{frame}

\begin{frame}[containsverbatim]
  \frametitle{Program L}
  With current methods we cannot prove termination, but dynamic evaluation terminates.
  How can we fix this?
\end{frame}

\begin{frame}[containsverbatim]
  \frametitle{Program L}
  With current methods we cannot prove termination, but dynamic evaluation terminates.
  How can we fix this?
  \\~\\
  \textcolor{red}{Introducing tuples!}
\end{frame}

\begin{frame}[containsverbatim]
  \frametitle{Program L}
\begin{lstlisting}[mathescape=true]
vars: [$x$]
requirements: {$x \geq 0$}
<!=_=!>
$y := 1;$
while ($x > 0$ || $y > 0$) ?{$true$} !{$\textcolor{red}{x , y}$} {
    if $x > 0$ {
        $x := x - 1;$
        $y := y + y;$
    } else {
        $y := y - 1;$
    };
};
#{$true$};
\end{lstlisting}
What is the meaning of this?
\end{frame}

\begin{frame}[containsverbatim]
  \frametitle{Program L}
\begin{align*}
  &\frac{
    \{I \land e \land \textcolor{blue}{v = \xi} \} s \{I \land \textcolor{blue}{v \prec \xi} \} \quad wf(\prec)
  }{
    \{I\} \texttt{ while } e \texttt{ invariant } I
          \texttt{ variant } \textcolor{blue}{v}, \prec \texttt{ do } s \{I \land \neg e\}
  }
\end{align*}
\\~\\
\\~\\
\\~\\
\\~\\
\\~\\
\\~\\
\\~\\
\\~\\
\\~\\
\\~\\
\\~\\
\\~\\
\\~\\
\\~\\
\\~\\
\\~\\
\\~\\
\\~\\
\\~\\
\end{frame}

\begin{frame}[containsverbatim]
  \frametitle{Program L}
\begin{align*}
  &\frac{
    \{I \land e \land \textcolor{blue}{v = \xi} \} s \{I \land \textcolor{blue}{v \prec \xi} \} \quad wf(\prec)
  }{
    \{I\} \texttt{ while } e \texttt{ invariant } I
          \texttt{ variant } \textcolor{blue}{v}, \prec \texttt{ do } s \{I \land \neg e\}
  }
\end{align*}
\begin{align*}
  \Downarrow &
\end{align*}
\begin{align*}
  &\frac{
    \{I \land e \land \textcolor{red}{u = \xi_{1} \land v = \xi_{2}} \} s \{I \land \textcolor{red}{(u \prec \xi_{1} \lor v \prec \xi_{2})} \} \quad wf(\prec)
  }{
    \{I\} \texttt{ while } e \texttt{ invariant } I
          \texttt{ variant } \textcolor{red}{u,v} , \prec \texttt{ do } s \{I \land \neg e\}
  }
\end{align*}
\\~\\
\\~\\
\\~\\
\\~\\
\\~\\
\\~\\
\\~\\
\\~\\
\\~\\
\\~\\
\\~\\
\\~\\
\\~\\
\end{frame}

\begin{frame}[containsverbatim]
  \frametitle{Program L}
\begin{align*}
  &\frac{
    \{I \land e \land \textcolor{blue}{v = \xi} \} s \{I \land \textcolor{blue}{v \prec \xi} \} \quad wf(\prec)
  }{
    \{I\} \texttt{ while } e \texttt{ invariant } I
          \texttt{ variant } \textcolor{blue}{v}, \prec \texttt{ do } s \{I \land \neg e\}
  }
\end{align*}
\begin{align*}
  \Downarrow &
\end{align*}
\begin{align*}
  &\frac{
    \{I \land e \land \textcolor{red}{u = \xi_{1} \land v = \xi_{2}} \} s \{I \land \textcolor{red}{(u \prec \xi_{1} \lor v \prec \xi_{2})} \} \quad wf(\prec)
  }{
    \{I\} \texttt{ while } e \texttt{ invariant } I
          \texttt{ variant } \textcolor{red}{u,v} , \prec \texttt{ do } s \{I \land \neg e\}
  }
\end{align*}
\begin{align*}
  \Downarrow &
\end{align*}
\begin{align*}
&WP(
    \texttt{while } e \texttt{ invariant } I
    \texttt{ variant } \textcolor{orange}{u,v}, \prec \texttt{ do } s
, Q)
=\\
&I \land \forall x_1, ..., x_k, \textcolor{orange}{\xi_{1}, \xi_{2}}\\
&\quad \quad (((e \land I \land \textcolor{orange}{\xi_{1} = u \land \xi_{2} = v}) \Rightarrow WP(s, I \land \textcolor{orange}{(u \prec \xi_{1} \lor v \prec \xi_{2})})) \\
&\quad \quad \quad \quad \land ((\neg e \land I) \Rightarrow Q)) [w_i \leftarrow x_i] \\
\end{align*}
\end{frame}

\begin{frame}[containsverbatim]
  \frametitle{Program L}
\begin{align*}
  &\frac{
    \{I \land e \land \textcolor{red}{u = \xi_{1} \land v = \xi_{2}} \} s \{I \land \textcolor{red}{(u \prec \xi_{1} \lor v \prec \xi_{2})} \} \quad wf(\prec)
  }{
    \{I\} \texttt{ while } e \texttt{ invariant } I
          \texttt{ variant } \textcolor{red}{u,v} , \prec \texttt{ do } s \{I \land \neg e\}
  }
\end{align*}
\begin{lstlisting}[escapeinside={(*}{*)}]
-- Updated AST ([] instead of Maybe)
(*While BExpr FOL \textcolor{red}{[Variant]} Stmt*)
\end{lstlisting}

\begin{lstlisting}[escapeinside={(*}{*)}]
-- Updated Parser
whileP :: Parser Stmt
whileP = do
  rword "while"
  c <- bExprP
  invs <- sepBy1 (symbol "?" >> local (const True) (cbrackets impP)) (symbol ";")
  let inv = foldr1 (./\.) invs
  (*\textcolor{red}{var <- option []}*)
        (*\textcolor{red}{(symbol "!" >> cbrackets (sepBy1 aExprP (symbol ",")))}*)
  While c inv var <$> cbrackets seqP
\end{lstlisting}
\end{frame}

\begin{frame}[containsverbatim]
  \frametitle{Program L}
\begin{align*}
  &\frac{
    \{I \land e \land \textcolor{red}{u = \xi_{1} \land v = \xi_{2}} \} s \{I \land \textcolor{red}{(u \prec \xi_{1} \lor v \prec \xi_{2})} \} \quad wf(\prec)
  }{
    \{I\} \texttt{ while } e \texttt{ invariant } I
          \texttt{ variant } \textcolor{red}{u,v} , \prec \texttt{ do } s \{I \land \neg e\}
  }
\end{align*}
\begin{lstlisting}[escapeinside={(*}{*)}]
wlp (While b inv vars s) q = do
  (quants, vars', veq) <- foldrM go
            ([], ftrue, ftrue) vars
  ...
  where
    ...
    go :: Variant -> ([FOL -> FOL], FOL, FOL)
               -> WP ([FOL -> FOL], FOL, FOL)
    go var (qs,rs,as) = do
        (quant, rel, ass) <- resolveVar var
        return (quant:qs, rel .\/. rs, ass ./\. as)
\end{lstlisting}
\end{frame}
\begin{frame}[containsverbatim]
  \frametitle{Program L}
  What about McCarthy 91?
  \\~\\
  Still not strong enough assertion language
\end{frame}
